\documentclass[fleqn,a4paper,12pt,titlepage,finall]{article}

\usepackage[russian]{babel}
\usepackage{tikz}

\usepackage{mathtools}
\usepackage{unicode-math}
\setmainfont{Fira Sans}
\setmonofont{Fira Math}
\setmathfont{Fira Math}

\everymath{\displaystyle \tt} % математически выражения

\linespread{1.15}
\usepackage[left=1.8cm,right=1.8cm,top=1.5cm,bottom=1.5cm]{geometry}

% Ссылки без прямоугольников
\usepackage[hidelinks]{hyperref}
\usepackage[all]{hypcap}

% Содержание
\usepackage{tocloft}
\renewcommand{\cftsecleader}{\cftdotfill{\cftdotsep}}
\renewcommand{\cftaftertoctitle}{\hfill}
\addto\captionsrussian{\renewcommand{\contentsname}{\hfill Содержание \hfill}}
\renewcommand{\cftsecfont}{}
\renewcommand{\cftsecpagefont}{} 

\setlength\parindent{0pt}
\setlength\mathindent{0pt}
\usepackage{lipsum}

\newcommand\vv[1]{\symbfit{#1}}

\usepackage{gensymb}

\begin{document}
{\huge \bf \centering Классическая механика \par}
\tableofcontents
\section{Кинематика материальной точки}
\subsection{Основные определения}
{\bf Кинематика} --- это раздел механики, изучающий движение тел без рассмотрения
причин этого движения. Задача кинематики --- математически точно описать
движение тела. \\
{\bf Материальная точка} --- это тело, размерами которого можно пренебречь.
Чтобы измерить расстояние, нужно сравнить его с длиной некоторого тела,
принятого за эталон. Чтобы измерить промежуток времени, нужно сравнить его с
продолжительностью некоторого процесса, принятого за эталон (например, с
колебанием маятника). Чтобы измерить любую физическую величину, нужно ввести
единицу измерения. \\
{\bf Метр} --- это расстояние, которое проходит свет в вакууме приблизительно за
$\frac{1}{3\cdot 10^8}$ секунды. \\
{\bf Секунда} --- это продолжительность приблизительно $10^{10}$ колебаний
электрона в атоме цезия.  \\
{\bf Ось координат} --- это прямая линия, на которой выбраны начало отсчёта,
положительное направление и единица измерения длины. \\
{\bf Радиус-вектор точки} --- это вектор, проведённый от начала отсчёта к данной
точке. \\
{\bf Орты декартовых координат} --- это единичные векторы, направленные вдоль
декартовых осей координат. \\
{\bf Проекция вектора на ось} --- это разность координат конца и начала вектора,
взятых по отношению к данной оси. \\
{\bf Перемещение} --- это разность радиус-векторов точки, взятых в два разных
момента времени.
\[\Delta \vv{r} = \vv{r_2} - \vv{r_1}\]
\subsection{Декартовы компоненты скорости и ускорения}
{\bf Скорость материальной точки} --- это отношение перемещения точки к длительности
перемещения в пределе, когда эта длительность стремится к нулю (производная по
времени).
\[\vv{v} = \lim_{\Delta t \to 0} \frac{\Delta \vv{r}}{\Delta t}	= \dot{r}\]
\[\vv{v} = \vv{i}v_x + \vv{j}v_y + \vv{k}v_z\]
\[\vv{v} = \frac{d\vv{r}}{dt} = \frac{d}{dt}(\vv{i}x+\vv{j}y+\vv{k}z) =
\vv{i}\dot{x} + \vv{j}\dot{y} + \vv{k}\dot{z}\]
\[|\vv{v}| = v = \sqrt{v_x^2 + v_y^2 + v_z^2}\]
{\bf Ускорение материальной точки} --- это производная скорости точки по времени.
\[\vv{a} = \frac{d\vv{v}}{dt} = \vv{\dot{v}} = \vv{\ddot{r}}\]
\[\vv{a} = \vv{i}a_x + \vv{j}a_y + \vv{k}a_z\]
\[\vv{a} = \frac{d\vv{v}}{dt} = \frac{d}{dt}(\vv{i}v_x+\vv{j}v_y+\vv{k}v_z) =
\vv{i}\ddot{x} + \vv{j}\ddot{y} + \vv{k}\ddot{z}\]
\[|\vv{a}| = a = \sqrt{a_x^2 + a_y^2 + a_z^2}\]
\subsection{Равномерное движение}
\[v_x = const = \frac{dx}{dt}\]
\[dx = v_xdt\]
\[\int_{x_0}^x dx = \int_0^t v_xdt\]
\[x-x_0 = v_xt\]
\[\boxed{x(t) = x_0 + v_xt}\]
\subsection{Равнопеременное движение}
\[a_x = const = \frac{dv_x}{dt}\]
\[dv_x = a_xdt\]
\[\int_{v_{x_0}}^{v_x} dv_x = \int_0^t a_xdt\]
\[v_x-v_{x_0} = a_xt\]
\[\boxed{v_x(t) = v_{x_0} + a_xt}\]
\[\int_{x_0}^x dx = \int_0^t v_xdt\]
\[x-x_0 = v_{x_0}t + \frac{a_xt^2}{2}\]
\[\boxed{x(t) = x_0 + v_{x_0}t + \frac{a_xt^2}{2}}\]
\subsection{Криволинейное движение}
{\bf Тангенциальное ускорение} --- это составляющая ускорения, параллельная
вектору скорости. \\
{\bf Нормальное ускорение} --- это составляющая ускорения, перпендикулярная
вектору скорости и направленная к центру кривизны траектории движения точки. \\
{\bf Круг кривизны кривой в точке} --- это круг, проходящий через данную точку
кривой $M$ и две другие точки кривой $N$ и $P$, лежащие по разные стороны от
$M$, в пределе при $N \to M$ и $P \to M$.
\[\vv{\tau} = \frac{\vv{v}}{v}, |\vv{\tau}| = 1, \vv{n} \perp \vv{\tau},
|\vv{n}| = 1\]
\[\vv{a} = \vv{\tau}a_{\tau} + \vv{n}a_n\]
\[\vv{a} = \dot{\vv{v}} = \frac{d}{dt}(\vv{\tau}v) = \vv{\tau}\frac{dv}{dt} +
v\frac{d\vv{\tau}}{dt}\]
\[d\vv{\tau} = \vv{n}\frac{dr}{R}\]
\[\frac{d\vv{\tau}}{dt} = \vv{n} \frac{dr}{Rdt} = \vv{n}\frac{v}{R}\]
\[\boxed{\vv{a} = \vv{\tau}\frac{dv}{dt} + \vv{n}\frac{v^2}{R}}\]
\[a = \sqrt{a_n^2 + a_{\tau}^2}\]
Найдём радиус кривизны
\[(x-x_c)^2 + (y-y_c)^2 = r^2\]
\[2(x-x_c) + 2(y-y_c)y' = 0 \text{ (Дифференцируем дважды по } x)\]
\[1+y'^2 + (y-y_c)y'' = 0\]
\[y-y_c = -\frac{1+y'^2}{y''}, x-x_c = \frac{1+y'^2}{y''}y'\]
\[\left(\frac{1+y'^2}{y''}y'\right)^2 + \left(\frac{1+y'^2}{y''}\right)^2 = R^2\]
\[\left(\frac{1+y'^2}{y''}\right)^2(1+y'^2) = R^2\]
\[\boxed{R=\frac{(1+y'^2)^{\frac{3}{2}}}{|y''|}}\]

\section{Относительность механического движения}
{\bf Относительность механического движения} --- это различие движения одного и
того же тела относительно разных тел (систем) отсчёта. \\
{\bf Поступательное движение} --- это движение, при котором направление осей не
меняется. \\
При поступательном движении подвижной системы отсчёта справедливы следующие
формулы:
\[\vv{r} = \vv{r_0} + \vv{r'}\]
\[\vv{v} = \vv{v_0} + \vv{v'}\]
\[\vv{a} = \vv{a_0} + \vv{a'}\]
Здесь $\vv{v}$ --- абсолютная скорость тела, ${\vv{v_0}}$ --- относительная
скорость тела в подвижной системе отсчёта, ${\vv{v'}}$ --- скорость системы.

\section{Принцип относительности. Преобразования Галилея и Лоренца}
\subsection{Принцип относительности Галилея}
Никакими механическими опытами, проведёнными внутри данной системы отсчёта,
нельзя установить, находится ли эта система в состоянии покоя или равномерно
прямолинейно движется. Иначе говоря, уравнения, выражающие физические законы,
должны быть инвариантны относительно преобразований, описывающих переход от
неподвижной системы отсчёта к системе, движущейся равномерно и прямолинейно. 
\subsection{Преобразования Галилея}
Рассмотрим неподвижную систему отсчёта ($x, y, z$) и систему, движущуюся
равномерно ($x', y', z', v$). Тогда преобразования Галилея выглядят так:
\[\begin{dcases}
	x = x' + vt \\
	y = y' \\
	z = z' \\
\end{dcases}\]
Как следствие получим правило сложения скоростей:
\[\begin{dcases}
	v_x = v_x' + v \\
	v_y = v_y' \\
	v_z = v_z' \\
\end{dcases}\]
\subsection{Гипотеза неподвижного эфира}
{\bf Гипотеза неподвижного эфира} --- это предположение о том, что скорость
света относительно Солнца равна $c = 3\cdot10^8$ м/с, а относительно Земли она
определяется правилом Галилея:
\[\begin{dcases}
	v_x^2 + v_y^2 = c^2 \\
	v_x = v_x' + v \\
	v_y = v_y' \\
\end{dcases}\]
{\bf Продольная скорость света} --- это скорость света относительно Земли в
направлении её движения по орбите.
\[v_{\parallel} = |v_x'| = c \pm v\]
{\bf Поперечная скорость света} --- это скорость света относительно Земли в
направлении, перпендикулярном её движению по орбите.
\[v_{\perp} = |v_y'| = \sqrt{c^2-v^2}\]
Продольная и поперечная скорости света не равны друг другу.\\
{\bf Интерференция света} --- взаимная компенсация действия света в некоторых
точках пространства ("свет + свет  = темнота"). \\

В 19 веке стало известно, что уравнения электромагнитного поля не инвариантны
относительно преобразований Галилея. Было решено проверить правило сложения
скоростей Галилея для электромагнитных волн. Мейкельсон решил использовать в
качестве подвижной системы отсчёта Землю в движении вокруг Солнца. Для
проведения опыта использовали интерферометр Мейкельсона, состоящего из двух
перпендикулярных зеркал, экрана и светоделительного зеркала.
\[\frac{l_1}{c-v} + \frac{l_1}{c+v} = \frac{2l_2}{\sqrt{c^2-v^2}}\]
После поворота на 90\textdegree:
\[\frac{l_2}{c-v} + \frac{l_2}{c+v} = \frac{2l_1}{\sqrt{c^2-v^2}} + \frac{T}{2}\]
Отсюда $l_1 \approx l_2 = \frac{1}{4}\lambda \frac{c^2}{v^2} \approx 10$ м.\\
Опыт показал, что повороты прибора не меняли наблюдаемую интерференционную
картину. Был сделан вывод, что гипотеза неподвижного эфира ошибочна ---
результат опыта был таким, как будто Земля неподвижна.
\subsection{Преобразования Лоренца}
{\bf Принцип постоянства скорости света}: скорость света не зависит от того,
по отношению к какой системе отсчёта (покоящейся или движущейся) она
определяется. \\
Преобразования Лоренца:
\[\begin{dcases}
	x = \frac{x' + vt'}{\sqrt{1 - \frac{v^2}{c^2}}} \\
	t = \frac{t' + \frac{x'v}{c^2}}{\sqrt{1 - \frac{v^2}{c^2}}} \\
	y = y' \\
	z = z' \\
\end{dcases}\]
Оказалось, что уравнения электромагнитного поля инвариантны относительно
преобразований Лоренца.\\
{\bf Принцип относительности Эйнштейна}: уравнения, выражающие физические
законы, должны быть инвариантны относительно преобразований Лоренца.
Как следствие можно получить правило сложения скоростей в теории
относительности:
\[v_x = \frac{dx}{dt}, v_x' = \frac{dx'}{dt'}\]
\[dx = \frac{dx' + vdt'}{\sqrt{1 - \frac{v^2}{c^2}}}\]
\[dt = \frac{dt' + \frac{dx'v}{c^2}}{\sqrt{1 - \frac{v^2}{c^2}}}\]
\[\boxed{v_x = \frac{v_x' + v}{1+\frac{v_x'v}{c^2}}}\]

\section{Кинематика твёрдого тела}
\subsection{Поступательное движение}
{\bf Твёрдое тело} --- это система материальных точек, расстояние между любой
парой которых неизменно. \\
{\bf Поступательное движение твёрдого тела} --- это движение, при котором
ориентация тела в пространстве сохраняется. \\
\[v_i = v\]
\subsection{Вращение вокруг оси}
{\bf Вращение твёрдого тела вокруг оси} --- это движение, при котором все точки
тела движутся по окружностям, а центры всех окружностей лежат на одной прямой,
называемой осью вращения. \\
\[v = \frac{dr}{dt} \approx \frac{dS}{dt}\]
{\bf Угол поворота тела} (в радианах) --- это отношение длины дуги окружности,
попадающей внутрь угла, к длине этой окружности.
\[\phi = \frac{S}{R}\]
\[\omega = \frac{d\varphi}{dt} = \dot{\varphi}\]
\[v \approx \frac{dS}{dt} = R\frac{d\varphi}{dt} = \omega R\]
{\bf Вектор угловой скорости} --- это вектор, направленный вдоль оси вращения по
правилу правого винта и равный по модулю производной угла по времени. \\
\subsection{Движение с одной неподвижной точкой}
{\bf Теорема Эйлера} --- движение тела с одной неподвижной точкой в каждый
момент времени можно рассматривать как движение вокруг некоторой неподвижной
оси, проходящей через точку закрепления --- мгновенной оси вращения.
\[\vv{v} = [\vv{\omega} \times \vv{r}]\]
\subsection{Положение тела в пространстве}
{\bf Матрица поворота тела} $S_{ij}$ --- это матрица, составленная из скалярных
произведений ортов двух координатных систем (неподвижной системы и системы,
связанной с телом).
\[S_{ij} = (\vv{e_i}, \vv{e_j})\]
Найдём преобразование координат при повороте тела
\[\vv{r} = \vv{e_1}x_1 + \vv{e_2}x_2 + \vv{e_3}x_3\]
\[\vv{r} = \vv{e_1'}x_1' + \vv{e_2'}x_2' + \vv{e_3'}x_3'\]
\[(\vv{e_1}, \vv{r}) = x_1 = x_1'(\vv{e_1}, \vv{e_1'}) + x_2'(\vv{e_1},
\vv{e_2'}) + x_3'(\vv{e_1}, \vv{e_3'})\]
\[\boxed{x_i = \sum_{j=1}^3 S_{ij}x_j'}\]
\section{Кинематика вращающихся систем отсчёта}
Какие особенности приобретают физические законы, если рассматривать их в системе
отсчёта, связанной с вращающимся телом? Как связаны между собой кинематические
характеристики точки в неподвижной и вращающейся системах?
\[\vv{r} = \vv{r_0} + \vv{r'}\]
\[d\vv{r} = d\vv{r_0} + d\vv{r'}\]
\[\vv{r'} = \vv{e_1'}x_1' + \vv{e_2'}x_2' + \vv{e_3'}x_3'\]
\[d\vv{r'} = \vv{e_1'}dx_1' + \vv{e_2'}dx_2' + \vv{e_3'}dx_3' + d\vv{e_1'}x_1' +
d\vv{e_2'}x_2' + d\vv{e_3'}x_3'\]
Здесь первая группа слагаемых характеризует изменение положения точки
относительно подвижной системы отсчёта, а вторая --- изменение положение
подвижной системы относительно неподвижной.
\[\vv{v}=[\vv{\omega}\times\vv{r}]\]
\[\vv{v} = \frac{d\vv{r}}{dt}\]
\[d\vv{r} = [\vv{\omega}\times\vv{r}]dt\]
\[d\vv{e_1'}x_1' + d\vv{e_2'}x_2' + d\vv{e_3'}x_3' =
[\vv{\omega}\times\vv{r'}]dt\]
\[d\vv{r'} = \vv{e_1'}dx_1' + \vv{e_2'}dx_2' + \vv{e_3'}dx_3' +
[\vv{\omega}\times\vv{r'}]dt\] 
\[d\vv{r} = d\vv{r_0} + d\vv{r'}\]
\[\boxed{\vv{v}  = \vv{v_0} + \vv{v'} + [\vv{\omega}\times\vv{r'}]}\]
\[d\vv{v}  = d\vv{v_0} + d\vv{v'} + [\vv{\omega}\times d\vv{r'}]\]
\[d\vv{v'} = \vv{e_1'}dv_1' + \vv{e_2'}dv_2' + \vv{e_3'}dv_3' +
[\vv{\omega}\times\vv{v'}]dt \text{ (получено аналогично }d\vv{r'})\]
\[[\vv{\omega}\times d\vv{r'}] = dt([\vv{\omega}\times \vv{v'}] +
[\vv{\omega}\times [\vv{\omega}\times \vv{r'}] ])\]
\[\boxed{\vv{a} = \vv{a_0} + \vv{a'} + 2[\vv{\omega}\times \vv{v'}] +
[\vv{\omega}\times [\vv{\omega}\times \vv{r'}]])}\]
\[\boxed{\vv{a} = \vv{a'} + \vv{a_{\text{п}}} + \vv{a_{\text{к}}}}, \text{ где }
\vv{a_{\text{п}}} = \vv{a_0} + [\vv{\omega}\times [\vv{\omega}\times
\vv{r'}]]\text{ (переносное}), \vv{a_{\text{к}}} = 2[\vv{\omega}\times
\vv{v'}]\text{ (кориолисово})\]

\section{Законы Ньютона}
\subsection{Основные определения}
{\bf Сила} --- это мера действия других тел на данное тело. \\
{\bf Масса тела} --- это мера отклика тела на действие силы. \\
{\bf Импульс} --- это произведение массы точки на её скорость. \\
{\bf Килограмм} --- масса эталонного тела, представляющего собой цилиндр из
сплава платины и иридия диаметром 39 мм и такой же высоты (определение
устарело).\\
{\bf 1 Ньютон} --- сила, вызывающая ускорение в 1 м/$с^2$ у тела массы 1 кг. \\
\subsection{Законы Ньютона}
{\bf Первый закон Ньютона}: всякое тело сохраняет состояние покоя или
равномерного прямолинейного движения до тех пор, пока другие тела не заставят
его изменить это состояние. \\
{\bf Второй закон Ньютона}: произведение массы материальной точки на ускорение
равно действующей на него силе. В импульсной формулировке: скорость изменения
импульса материальной точки равна действующей на неё силе.\\
\[\boxed{\vv{F} = m\vv{a}}\]
\[\vv{p} = m\vv{v}\]
\[\dot{\vv{p}} = m\vv{a}\]
\[\boxed{\dot{\vv{p}} = \vv{F}}\]
Второй закон Ньютона не выполняется в двух случаях: тело движется со скоростью,
близкой к скорости света, либо тело очень мало и движется в малой области
пространства. \\
{\bf Третий закон Ньютона}: действия двух тел друг на друга равны по модулю и
противоположно направлены.
\[\boxed{\vv{F_{12} = -\vv{F_{21}}}}\]
Силы взаимодействия приложены к разным телам, направлены вдоль одной прямой и
имеют одинаковую природу.\\
Если на материальную точку одновременно действуют несколько сил, то оно движется
так, как если бы на него действовала одна сила, равная их векторной сумме.

\section{Силы в механике}
\subsection{Гравитационные силы}
{\bf Закон всемирного тяготения}: любые две частицы притягиваются друг к другу с
силой, пропорциональной их массам и обратно пропорциональной квадрату расстояния
между ними.
\[F = G\frac{m_1m_2}{R^2}, \text{ где } G \approx
6.67\cdot10^{-11}\frac{\text{м}^3}{\text{кг}\cdot c^2}\]
{\bf Принцип суперпозиции}: каждая пара частиц взаимодействует независимо, т.е.
так, как будто других частиц нет. Например, при притяжении материальной точки к
однородному шару сила такова, как если бы вся масса шара находилась в его
центре.\\
{\bf Масса Земли}:
\[mg = G\frac{Mm}{R^2}\]
\[M = \frac{gR^2}{G} \approx 5.97 \cdot 10^{24} \text{кг}\]
{\bf Период вращения Луны}:
\[m\frac{v^2}{r} = G\frac{Mm}{r^2}\]
\[v^2=G\frac{M}{r}\]
\[T = \frac{2\pi r}{v} = 2\pi \frac{r\sqrt{r}}{\sqrt{GM}}\]
\[M = \frac{gR^2}{G}\]
\[T = 2\pi \frac{r\sqrt{r}}{R\sqrt{g}} \approx 30 \text{ суток}\]
\subsection{Сила упругости}
{\bf Упругое тело} --- это тело, которое восстанавливает свою форму после
прекращения действия силы. \\
{\bf Закон Гука}: сила упругости пропорциональна величине деформации. Это
приближённое выражение, верное при малых деформациях.\\
\[\vv{F_x} = -k\vv{x}\]
\subsection{Сила трения}
{\bf Сила нормального давления (реакции опоры)} --- это составляющая силы
взаимодействия соприкасающихся тел, перпендикулярная поверхности
соприкосновения.
{\bf Трение покоя} --- это трение, возникающее при отсутствии движения
соприкасающихся тел. \\
\[\vv{F_{\text{тр.п.}}} = -\vv{F_{\text{внеш.}}}\]
{\bf Трение скольжения} --- это трение, возникающее при скольжении одного тела
по поверхности другого. Опыт показывает, что сила трения скольжения примерно
равна максимальной силе трения покоя. \\
\[F_{\text{тр.ск.}} = \mu N \approx F_{max \text{ тр.п.}}\]
{\bf Вязкое трение (сопротивление)} --- это трение, препятствующее движению тела
в сплошной среде. Сила вязкого трения пропорциональна скорости движения.
\[F_{\text{в.тр.}} = kv\]
\subsection{Электромагнитные силы}
{\bf Электрический заряд} --- это метра электрического взаимодействия тела. \\
{\bf Электрическое поле} --- это поле, созданное электрическими зарядами и
проявляющее себя действием на электрические заряды. \\
{\bf Напряжённость поля} --- это мера действия электрического поля на заряд. \\
\[\vv{E} = \frac{\vv{F}}{q}\]
{\bf Сила Кулона} --- это сила взаимодействия двух точечных зарядов в вакууме.
\\
\[\vv{F_q} = q\vv{E}\]
{\bf Электрический ток} --- это направленное движение заряженных частиц под
воздействием электрического поля. \\
{\bf Магнитное поле} --- это поле, созданное электрическим током и проявляющее
себя действием на движущиеся электрические заряды. \\
{\bf Магнитная индукция} --- это мера действия магнитного поля на заряд. \\
{\bf Электромагнитное поле} --- это поле, образованное электрическим и магнитным
полями, направленными перпендикулярно друг другу. \\
{\bf Сила Лоренца} --- это сила, с которое электромагнитное поле действует
движущийся точечный заряд.\\
\[F_L = q[\vv{v}\times \vv{B}]\]
\subsection{Релятивистское уравнение движения}
Обобщим второй закон Ньютона на случай движения тел с большими скоростями. Для
этого введём сопровождающую систему отсчёта, в которой выполняется второй закон
Ньютона, далее перейдём к неподвижной системе отсчёта с осями координат,
параллельными осям сопровождающей системы (используем преобразования Лоренца), а
затем поворачиваем неподвижную систему отсчёта.
\[\dot{\vv{p}} = \vv{F}\]
\[\vv{p} = \frac{m\vv{v}}{\sqrt{1-\frac{v^2}{c^2}}}\]
\section{Неинерциальные системы отсчёта. Сила инерции}
{\bf Инерциальная система отсчёта} --- это такая система, в которой любое тело,
бесконечно удалённое от других тел, не испытывает ускорения. Систему отсчёта,
связанную с Землёй, обычно можно считать инерциальной. Неинерциальными  являются
системы отсчёта, движущиеся с большим ускорением относительно Земли.\\
{\bf Сила инерции} --- добавочная сила, действующая на материальную точку в
неинерциальной системе отсчёта. Сила инерции отлична от нуля только для
наблюдателя, связанного с неинерциальной системой отсчёта, и не подчиняется
третьему закону Ньютона.\\
\[\vv{F} = m\vv{a}\]
\[\vv{F} = m\vv{a} + m\vv{a'} - m\vv{a'}\]
\[m\vv{a'} = \vv{F} - m(\vv{a} - \vv{a'})\]
\[\boxed{\vv{F_{\text{ин}}} = -m(\vv{a} - \vv{a'})}\]
\[\vv{a} = \vv{a'} + \vv{a_{\text{п}}} + \vv{a_{\text{к}}}\]
\[\vv{F_{\text{ин}}} = \vv{F_{\text{п}}} + \vv{F_{\text{к}}} =
-m\vv{a_0}+m\omega^2\vv{r} - 2m[\vv{\omega}\times \vv{v'}]\]
Второе слагаемое в этой сумме называется {\bf центробежной силой}, а третье ---
{\bf Кориолисовой силой}. \\
На тела, движущиеся в северном полушарии, действует Кориолиса, направленная
вправо относительно движения. Например, плоскость колебаний маятника Фуко
медленно поворачивается за счёт силы Кориолиса. Этот опыт доказывает вращение
Земли.\\
{\bf Невесомость} --- это исчезновение веса тела, вызванное ускорением системы
отсчёта. \\
{\bf Перегрузка} --- это возрастание веса тела, вызванное ускорением системы
отсчёта. \\
{\bf Центрифуга} --- это устройство, использующее центробежную силу инерции. \\

\section{Импульс системы частиц. Движение центра масс}
\subsection{Основные определения}
{\bf Импульс системы частиц} --- это сумма импульсов отдельных частиц системы.\\
\[\vv{p} = \sum_i\vv{p_i} = \sum_im_i\vv{v_i}\]
{\bf Центр масс системы частиц} --- это точка, радиус-вектор которой
определяется формулой:
\[\vv{r_c} = \frac{1}{m}\sum_im_i\vv{r_i}, \text{ где } m = \sum_im_i\]
Для однородных и симметричных тел центр масс совпадает с геометрическим центром.
В качестве примере рассмотрим систему из двух одинаковых точек.
\[\vv{r_c} = \frac{1}{2m}(\vv{r_1}m + \vv{r_2}m) = \frac{\vv{r_1}+\vv{r_2}}{2}\]
\subsection{Движение центра масс}
Импульс тела зависит от скорости центра масс
\[\begin{dcases}
	\vv{v_c} = \dot{r_c} = \frac{1}{m}\sum_im_i\vv{v_i} \\
	\vv{a_c} = \dot{v_c} = \frac{1}{m}\sum_im_i\vv{a_i} \\
\end{dcases}\]
\[\vv{p} = \sum_im_i\vv{v_i} = m\vv{v_c}\]

{\bf Внутренние силы} --- это силы взаимодействия между телами данной системы.
\\
\[\vv{f_{ij}} \text{ --- сила, действующая на i со стороны j }\]
{\bf Внешние силы} --- это силы, действующие на тела системы, со стороны тел, не
входящих в данную систему. \\
\[\vv{F_i} \text{ --- сила, действующая на i}\]
Просуммируем все силы
\[\sum_im_i\vv{a_i} = \sum_{i,j}\vv{f_{ij}} + \sum_i\vv{F_i}\]
По третьему закону Ньютона $\vv{f_{ij} = -\vv{f_{ji}}}$, то есть сумма
внутренних сил для любой пары частиц равна нулю. Следовательно, сумма всех
внутренних сил системы равна нулю
\[\sum_im_i\vv{a_i} = \sum_i\vv{F_i} = m\vv{a_c}\]
\[\boxed{m\vv{a_c} = \vv{F_{\text{внеш}}}}\]
Центр масс движется так, как если бы в нём находилась вся масса системы и к ней
были бы приложены все внешние силы.

\section{Закон сохранения импульса}
\subsection{Законы сохранения и изменения импульса}
{\bf Закон сохранения импульса}: если сумма внешних сил равна нулю, то импульс
системы сохраняется. \\
{\bf Закон изменения импульса}: изменение импульса равно сумме внешних сил,
действующих на систему. \\
\[\vv{p} = \sum_im_i\vv{v_i}\]
\[\dot{\vv{p}} = \sum_im_i\vv{a_i} = m\vv{a_c} =\vv{F_{\text{внеш}}} \]
{\bf Закон сохранения импульса}: если существует ось, проекция внешних сил на
которую равна нулю, то импульс системы относительно этой оси сохраняется. \\
\subsection{Реактивное движение}
Введём следующие обозначения: $\vv{v}$ --- скорость ракеты относительно Земли,
$\vv{u}$ --- скорость газов относительно Земли, $\vv{c}$ --- скорость газов
относительно ракеты, $\mu$ --- удельный расход топлива, $m$ --- масса ракеты с
топливом, $dm$ --- масса сгорающего топлива.
\[\begin{dcases}
	\vv{u} = \vv{v} + \vv{c} \\
	m\vv{v} = (m-dm)(\vv{v}+d\vv{v})+\vv{u}dm \\
\end{dcases}\]
\[md\vv{v} = -\vv{c}dm\]
\[m\vv{a} = -\vv{c}\frac{dm}{dt} = -\vv{c}\mu\]
\[\boxed{\vv{F} = -\mu\vv{c}}\]
\[m\frac{dv}[dt] = \mu c\]
\[\frac{dv}{c} = \frac{\mu dt}{m}\]
\[m(t) = m_0 - \mu t\]
\[\int_0^v\frac{dv}{c} = \int_0^t\frac{\mu dt}{m0-\mu t}\]
\[\boxed{\frac{v}{c} = \ln\frac{m_0}{m}}\]

\section{Работа и потенциальная энергия}

\end{document}
