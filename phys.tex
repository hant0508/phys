\documentclass[fleqn,a4paper,12pt,titlepage,finall]{article}

\usepackage[russian]{babel}
\usepackage{tikz}

\usepackage{mathtools}
\usepackage{unicode-math}
\setmainfont{Fira Sans}
\setmonofont{Fira Math}
\setmathfont{Fira Math}

\everymath{\displaystyle \tt} % математически выражения

\linespread{1.15}
\usepackage[left=1.8cm,right=1.8cm,top=1.5cm,bottom=1.5cm]{geometry}

% Ссылки без прямоугольников
\usepackage[hidelinks]{hyperref}
\usepackage[all]{hypcap}

% Содержание
\usepackage{tocloft}
\renewcommand{\cftsecleader}{\cftdotfill{\cftdotsep}}
\renewcommand{\cftaftertoctitle}{\hfill}
\addto\captionsrussian{\renewcommand{\contentsname}{\hfill Содержание \hfill}}
\renewcommand{\cftsecfont}{}
\renewcommand{\cftsecpagefont}{} 

\setlength\parindent{0pt}
\setlength\mathindent{0pt}
\usepackage{lipsum}

\newcommand\vv[1]{\symbfit{#1}}

\usepackage{gensymb}

\begin{document}
{\huge \bf \centering Классическая механика \par}
\tableofcontents
\section{Кинематика материальной точки}
\subsection{Основные определения}
{\bf Кинематика} --- это раздел механики, изучающий движение тел без рассмотрения
причин этого движения. Задача кинематики --- математически точно описать
движение тела. \\
{\bf Материальная точка} --- это тело, размерами которого можно пренебречь.
Чтобы измерить расстояние, нужно сравнить его с длиной некоторого тела,
принятого за эталон. Чтобы измерить промежуток времени, нужно сравнить его с
продолжительностью некоторого процесса, принятого за эталон (например, с
колебанием маятника). Чтобы измерить любую физическую величину, нужно ввести
единицу измерения. \\
{\bf Метр} --- это расстояние, которое проходит свет в вакууме приблизительно за
$\frac{1}{3\cdot 10^8}$ секунды. \\
{\bf Секунда} --- это продолжительность приблизительно $10^{10}$ колебаний
электрона в атоме цезия.  \\
{\bf Ось координат} --- это прямая линия, на которой выбраны начало отсчёта,
положительное направление и единица измерения длины. \\
{\bf Радиус-вектор точки} --- это вектор, проведённый от начала отсчёта к данной
точке. \\
{\bf Орты декартовых координат} --- это единичные векторы, направленные вдоль
декартовых осей координат. \\
{\bf Проекция вектора на ось} --- это разность координат конца и начала вектора,
взятых по отношению к данной оси. \\
{\bf Перемещение} --- это разность радиус-векторов точки, взятых в два разных
момента времени.
\[\Delta \vv{r} = \vv{r_2} - \vv{r_1}\]
\subsection{Декартовы компоненты скорости и ускорения}
{\bf Скорость материальной точки} --- это отношение перемещения точки к длительности
перемещения в пределе, когда эта длительность стремится к нулю (производная по
времени).
\[\vv{v} = \lim_{\Delta t \to 0} \frac{\Delta \vv{r}}{\Delta t}	= \dot{r}\]
\[\vv{v} = \vv{i}v_x + \vv{j}v_y + \vv{k}v_z\]
\[\vv{v} = \frac{d\vv{r}}{dt} = \frac{d}{dt}(\vv{i}x+\vv{j}y+\vv{k}z) =
\vv{i}\dot{x} + \vv{j}\dot{y} + \vv{k}\dot{z}\]
\[|\vv{v}| = v = \sqrt{v_x^2 + v_y^2 + v_z^2}\]
{\bf Ускорение материальной точки} --- это производная скорости точки по времени.
\[\vv{a} = \frac{d\vv{v}}{dt} = \vv{\dot{v}} = \vv{\ddot{r}}\]
\[\vv{a} = \vv{i}a_x + \vv{j}a_y + \vv{k}a_z\]
\[\vv{a} = \frac{d\vv{v}}{dt} =
\frac{d}{dt}\left(\vv{i}v_x+\vv{j}v_y+\vv{k}v_z\right) =
\vv{i}\ddot{x} + \vv{j}\ddot{y} + \vv{k}\ddot{z}\]
\[|\vv{a}| = a = \sqrt{a_x^2 + a_y^2 + a_z^2}\]
\subsection{Равномерное движение}
\[v_x = const = \frac{dx}{dt}\]
\[dx = v_xdt\]
\[\int_{x_0}^x dx = \int_0^t v_xdt\]
\[x-x_0 = v_xt\]
\[\boxed{x(t) = x_0 + v_xt}\]
\subsection{Равнопеременное движение}
\[a_x = const = \frac{dv_x}{dt}\]
\[dv_x = a_xdt\]
\[\int_{v_{x_0}}^{v_x} dv_x = \int_0^t a_xdt\]
\[v_x-v_{x_0} = a_xt\]
\[\boxed{v_x(t) = v_{x_0} + a_xt}\]
\[\int_{x_0}^x dx = \int_0^t v_xdt\]
\[x-x_0 = v_{x_0}t + \frac{a_xt^2}{2}\]
\[\boxed{x(t) = x_0 + v_{x_0}t + \frac{a_xt^2}{2}}\]
\subsection{Криволинейное движение}
{\bf Тангенциальное ускорение} --- это составляющая ускорения, параллельная
вектору скорости. \\
{\bf Нормальное ускорение} --- это составляющая ускорения, перпендикулярная
вектору скорости и направленная к центру кривизны траектории движения точки. \\
{\bf Круг кривизны кривой в точке} --- это круг, проходящий через данную точку
кривой $M$ и две другие точки кривой $N$ и $P$, лежащие по разные стороны от
$M$, в пределе при $N \to M$ и $P \to M$.
\[\vv{\tau} = \frac{\vv{v}}{v}, |\vv{\tau}| = 1, \vv{n} \perp \vv{\tau},
|\vv{n}| = 1\]
\[\vv{a} = \vv{\tau}a_{\tau} + \vv{n}a_n\]
\[\vv{a} = \dot{\vv{v}} = \frac{d}{dt}(\vv{\tau}v) = \vv{\tau}\frac{dv}{dt} +
v\frac{d\vv{\tau}}{dt}\]
\[d\vv{\tau} = \vv{n}\frac{dr}{R}\]
\[\frac{d\vv{\tau}}{dt} = \vv{n} \frac{dr}{Rdt} = \vv{n}\frac{v}{R}\]
\[\boxed{\vv{a} = \vv{\tau}\frac{dv}{dt} + \vv{n}\frac{v^2}{R}}\]
\[a = \sqrt{a_n^2 + a_{\tau}^2}\]
Найдём радиус кривизны
\[(x-x_c)^2 + (y-y_c)^2 = r^2\]
\[2(x-x_c) + 2(y-y_c)y' = 0 \text{ (Дифференцируем дважды по } x)\]
\[1+y'^2 + (y-y_c)y'' = 0\]
\[y-y_c = -\frac{1+y'^2}{y''}, x-x_c = \frac{1+y'^2}{y''}y'\]
\[\left(\frac{1+y'^2}{y''}y'\right)^2 + \left(\frac{1+y'^2}{y''}\right)^2 = R^2\]
\[\left(\frac{1+y'^2}{y''}\right)^2(1+y'^2) = R^2\]
\[\boxed{R=\frac{(1+y'^2)^{\frac{3}{2}}}{|y''|}}\]

\section{Относительность механического движения}
{\bf Относительность механического движения} --- это различие движения одного и
того же тела относительно разных тел (систем) отсчёта. \\
{\bf Поступательное движение} --- это движение, при котором направление осей не
меняется. \\
При поступательном движении подвижной системы отсчёта справедливы следующие
формулы:
\[\vv{r} = \vv{r_0} + \vv{r'}\]
\[\vv{v} = \vv{v_0} + \vv{v'}\]
\[\vv{a} = \vv{a_0} + \vv{a'}\]
Здесь $\vv{v}$ --- абсолютная скорость тела, ${\vv{v_0}}$ --- относительная
скорость тела в подвижной системе отсчёта, ${\vv{v'}}$ --- скорость системы.

\section{Принцип относительности. Преобразования Галилея и Лоренца}
\subsection{Принцип относительности Галилея}
Никакими механическими опытами, проведёнными внутри данной системы отсчёта,
нельзя установить, находится ли эта система в состоянии покоя или равномерно
прямолинейно движется. Иначе говоря, уравнения, выражающие физические законы,
должны быть инвариантны относительно преобразований, описывающих переход от
неподвижной системы отсчёта к системе, движущейся равномерно и прямолинейно. 
\subsection{Преобразования Галилея}
Рассмотрим неподвижную систему отсчёта ($x, y, z$) и систему, движущуюся
равномерно ($x', y', z', v$). Тогда преобразования Галилея выглядят так:
\[\begin{dcases}
	x = x' + vt \\
	y = y' \\
	z = z' \\
\end{dcases}\]
Как следствие получим правило сложения скоростей:
\[\begin{dcases}
	v_x = v_x' + v \\
	v_y = v_y' \\
	v_z = v_z' \\
\end{dcases}\]
\subsection{Гипотеза неподвижного эфира}
{\bf Гипотеза неподвижного эфира} --- это предположение о том, что скорость
света относительно Солнца равна $c = 3\cdot10^8$ м/с, а относительно Земли она
определяется правилом Галилея:
\[\begin{dcases}
	v_x^2 + v_y^2 = c^2 \\
	v_x = v_x' + v \\
	v_y = v_y' \\
\end{dcases}\]
{\bf Продольная скорость света} --- это скорость света относительно Земли в
направлении её движения по орбите.
\[v_{\parallel} = |v_x'| = c \pm v\]
{\bf Поперечная скорость света} --- это скорость света относительно Земли в
направлении, перпендикулярном её движению по орбите.
\[v_{\perp} = |v_y'| = \sqrt{c^2-v^2}\]
Продольная и поперечная скорости света не равны друг другу.\\
{\bf Интерференция света} --- взаимная компенсация действия света в некоторых
точках пространства ("свет + свет  = темнота"). \\

В 19 веке стало известно, что уравнения электромагнитного поля не инвариантны
относительно преобразований Галилея. Было решено проверить правило сложения
скоростей Галилея для электромагнитных волн. Мейкельсон решил использовать в
качестве подвижной системы отсчёта Землю в движении вокруг Солнца. Для
проведения опыта использовали интерферометр Мейкельсона, состоящего из двух
перпендикулярных зеркал, экрана и светоделительного зеркала.
\[\frac{l_1}{c-v} + \frac{l_1}{c+v} = \frac{2l_2}{\sqrt{c^2-v^2}}\]
После поворота на 90\textdegree:
\[\frac{l_2}{c-v} + \frac{l_2}{c+v} = \frac{2l_1}{\sqrt{c^2-v^2}} + \frac{T}{2}\]
Отсюда $l_1 \approx l_2 = \frac{1}{4}\lambda \frac{c^2}{v^2} \approx 10$ м.\\
Опыт показал, что повороты прибора не меняли наблюдаемую интерференционную
картину. Был сделан вывод, что гипотеза неподвижного эфира ошибочна ---
результат опыта был таким, как будто Земля неподвижна.
\subsection{Преобразования Лоренца}
{\bf Принцип постоянства скорости света}: скорость света не зависит от того,
по отношению к какой системе отсчёта (покоящейся или движущейся) она
определяется. \\
Преобразования Лоренца:
\[\begin{dcases}
	x = \frac{x' + vt'}{\sqrt{1 - \frac{v^2}{c^2}}} \\
	t = \frac{t' + \frac{x'v}{c^2}}{\sqrt{1 - \frac{v^2}{c^2}}} \\
	y = y' \\
	z = z' \\
\end{dcases}\]
Оказалось, что уравнения электромагнитного поля инвариантны относительно
преобразований Лоренца.\\
{\bf Принцип относительности Эйнштейна}: уравнения, выражающие физические
законы, должны быть инвариантны относительно преобразований Лоренца.
Как следствие можно получить правило сложения скоростей в теории
относительности:
\[v_x = \frac{dx}{dt}, v_x' = \frac{dx'}{dt'}\]
\[dx = \frac{dx' + vdt'}{\sqrt{1 - \frac{v^2}{c^2}}}\]
\[dt = \frac{dt' + \frac{dx'v}{c^2}}{\sqrt{1 - \frac{v^2}{c^2}}}\]
\[\boxed{v_x = \frac{v_x' + v}{1+\frac{v_x'v}{c^2}}}\]

\section{Кинематика твёрдого тела}
\subsection{Поступательное движение}
{\bf Твёрдое тело} --- это система материальных точек, расстояние между любой
парой которых неизменно. \\
{\bf Поступательное движение твёрдого тела} --- это движение, при котором
ориентация тела в пространстве сохраняется. \\
\[v_i = v\]
\subsection{Вращение вокруг оси}
{\bf Вращение твёрдого тела вокруг оси} --- это движение, при котором все точки
тела движутся по окружностям, а центры всех окружностей лежат на одной прямой,
называемой осью вращения. \\
\[v = \frac{dr}{dt} \approx \frac{dS}{dt}\]
{\bf Угол поворота тела} (в радианах) --- это отношение длины дуги окружности,
попадающей внутрь угла, к длине этой окружности.
\[\phi = \frac{S}{R}\]
\[\omega = \frac{d\varphi}{dt} = \dot{\varphi}\]
\[v \approx \frac{dS}{dt} = R\frac{d\varphi}{dt} = \omega R\]
{\bf Вектор угловой скорости} --- это вектор, направленный вдоль оси вращения по
правилу правого винта и равный по модулю производной угла по времени. \\
\subsection{Движение с одной неподвижной точкой}
{\bf Теорема Эйлера} --- движение тела с одной неподвижной точкой в каждый
момент времени можно рассматривать как движение вокруг некоторой неподвижной
оси, проходящей через точку закрепления --- мгновенной оси вращения.
\[\vv{v} = [\vv{\omega} \times \vv{r}]\]
\subsection{Положение тела в пространстве}
{\bf Матрица поворота тела} $S_{ij}$ --- это матрица, составленная из скалярных
произведений ортов двух координатных систем (неподвижной системы и системы,
связанной с телом).
\[S_{ij} = \left(\vv{e_i}, \vv{e_j}\right)\]
Найдём преобразование координат при повороте тела
\[\vv{r} = \vv{e_1}x_1 + \vv{e_2}x_2 + \vv{e_3}x_3\]
\[\vv{r} = \vv{e_1'}x_1' + \vv{e_2'}x_2' + \vv{e_3'}x_3'\]
\[(\vv{e_1}, \vv{r}) = x_1 = x_1'(\vv{e_1}, \vv{e_1'}) + x_2'(\vv{e_1},
\vv{e_2'}) + x_3'(\vv{e_1}, \vv{e_3'})\]
\[\boxed{x_i = \sum_{j=1}^3 S_{ij}x_j'}\]
\section{Кинематика вращающихся систем отсчёта}
Какие особенности приобретают физические законы, если рассматривать их в системе
отсчёта, связанной с вращающимся телом? Как связаны между собой кинематические
характеристики точки в неподвижной и вращающейся системах?
\[\vv{r} = \vv{r_0} + \vv{r'}\]
\[d\vv{r} = d\vv{r_0} + d\vv{r'}\]
\[\vv{r'} = \vv{e_1'}x_1' + \vv{e_2'}x_2' + \vv{e_3'}x_3'\]
\[d\vv{r'} = \vv{e_1'}dx_1' + \vv{e_2'}dx_2' + \vv{e_3'}dx_3' + d\vv{e_1'}x_1' +
d\vv{e_2'}x_2' + d\vv{e_3'}x_3'\]
Здесь первая группа слагаемых характеризует изменение положения точки
относительно подвижной системы отсчёта, а вторая --- изменение положение
подвижной системы относительно неподвижной.
\[\vv{v}=[\vv{\omega}\times\vv{r}]\]
\[\vv{v} = \frac{d\vv{r}}{dt}\]
\[d\vv{r} = [\vv{\omega}\times\vv{r}]dt\]
\[d\vv{e_1'}x_1' + d\vv{e_2'}x_2' + d\vv{e_3'}x_3' =
[\vv{\omega}\times\vv{r'}]dt\]
\[d\vv{r'} = \vv{e_1'}dx_1' + \vv{e_2'}dx_2' + \vv{e_3'}dx_3' +
[\vv{\omega}\times\vv{r'}]dt\] 
\[d\vv{r} = d\vv{r_0} + d\vv{r'}\]
\[\boxed{\vv{v}  = \vv{v_0} + \vv{v'} + [\vv{\omega}\times\vv{r'}]}\]
\[d\vv{v}  = d\vv{v_0} + d\vv{v'} + [\vv{\omega}\times d\vv{r'}]\]
\[d\vv{v'} = \vv{e_1'}dv_1' + \vv{e_2'}dv_2' + \vv{e_3'}dv_3' +
[\vv{\omega}\times\vv{v'}]dt \text{ (получено аналогично }d\vv{r'})\]
\[[\vv{\omega}\times d\vv{r'}] = dt([\vv{\omega}\times \vv{v'}] +
[\vv{\omega}\times [\vv{\omega}\times \vv{r'}] ])\]
\[\boxed{\vv{a} = \vv{a_0} + \vv{a'} + 2[\vv{\omega}\times \vv{v'}] +
\left[\vv{\omega}\times [\vv{\omega}\times \vv{r'}]\right]}\]
\[\boxed{\vv{a} = \vv{a'} + \vv{a_{\text{п}}} + \vv{a_{\text{к}}}}, \text{ где }
\vv{a_{\text{п}}} = \vv{a_0} + \left[\vv{\omega}\times [\vv{\omega}\times
\vv{r'}]\right]\text{ (переносное}), \vv{a_{\text{к}}} = 2[\vv{\omega}\times
\vv{v'}]\text{ (кориолисово})\]

\section{Законы Ньютона}
\subsection{Основные определения}
{\bf Сила} --- это мера действия других тел на данное тело. \\
{\bf Масса тела} --- это мера отклика тела на действие силы. \\
{\bf Импульс} --- это произведение массы точки на её скорость. \\
{\bf Килограмм} --- масса эталонного тела, представляющего собой цилиндр из
сплава платины и иридия диаметром 39 мм и такой же высоты (определение
устарело).\\
{\bf 1 Ньютон} --- сила, вызывающая ускорение в 1 м/$с^2$ у тела массы 1 кг. \\
\subsection{Законы Ньютона}
{\bf Первый закон Ньютона}: всякое тело сохраняет состояние покоя или
равномерного прямолинейного движения до тех пор, пока другие тела не заставят
его изменить это состояние. \\
{\bf Второй закон Ньютона}: произведение массы материальной точки на ускорение
равно действующей на него силе. В импульсной формулировке: скорость изменения
импульса материальной точки равна действующей на неё силе.\\
\[\boxed{\vv{F} = m\vv{a}}\]
\[\vv{p} = m\vv{v}\]
\[\dot{\vv{p}} = m\vv{a}\]
\[\boxed{\dot{\vv{p}} = \vv{F}}\]
Второй закон Ньютона не выполняется в двух случаях: тело движется со скоростью,
близкой к скорости света, либо тело очень мало и движется в малой области
пространства. \\
{\bf Третий закон Ньютона}: действия двух тел друг на друга равны по модулю и
противоположно направлены.
\[\boxed{\vv{F_{12} = -\vv{F_{21}}}}\]
Силы взаимодействия приложены к разным телам, направлены вдоль одной прямой и
имеют одинаковую природу.\\
Если на материальную точку одновременно действуют несколько сил, то она движется
так, как если бы на неё действовала одна сила, равная их векторной сумме.

\section{Силы в механике}
\subsection{Гравитационные силы}
{\bf Закон всемирного тяготения}: любые две частицы притягиваются друг к другу с
силой, пропорциональной их массам и обратно пропорциональной квадрату расстояния
между ними.
\[F = G\frac{m_1m_2}{R^2}, \text{ где } G \approx
6.67\cdot10^{-11}\frac{\text{м}^3}{\text{кг}\cdot c^2}\]
{\bf Принцип суперпозиции}: каждая пара частиц взаимодействует независимо, т.е.
так, как будто других частиц нет. Например, при притяжении материальной точки к
однородному шару сила такова, как если бы вся масса шара находилась в его
центре.\\
{\bf Масса Земли}:
\[mg = G\frac{Mm}{R^2}\]
\[M = \frac{gR^2}{G} \approx 5.97 \cdot 10^{24} \text{кг}\]
{\bf Период вращения Луны}:
\[m\frac{v^2}{r} = G\frac{Mm}{r^2}\]
\[v^2=G\frac{M}{r}\]
\[T = \frac{2\pi r}{v} = 2\pi \frac{r\sqrt{r}}{\sqrt{GM}}\]
\[M = \frac{gR^2}{G}\]
\[T = 2\pi \frac{r\sqrt{r}}{R\sqrt{g}} \approx 30 \text{ суток}\]
\subsection{Сила упругости}
{\bf Упругое тело} --- это тело, которое восстанавливает свою форму после
прекращения действия силы. \\
{\bf Закон Гука}: сила упругости пропорциональна величине деформации. Это
приближённое выражение, верное при малых деформациях.\\
\[\vv{F_x} = -k\vv{x}\]
\subsection{Сила трения}
{\bf Сила нормального давления (реакции опоры)} --- это составляющая силы
взаимодействия соприкасающихся тел, перпендикулярная поверхности
соприкосновения.
{\bf Трение покоя} --- это трение, возникающее при отсутствии движения
соприкасающихся тел. \\
\[\vv{F_{\text{тр.п.}}} = -\vv{F_{\text{внеш.}}}\]
{\bf Трение скольжения} --- это трение, возникающее при скольжении одного тела
по поверхности другого. Опыт показывает, что сила трения скольжения примерно
равна максимальной силе трения покоя. \\
\[F_{\text{тр.ск.}} = \mu N \approx F_{max \text{ тр.п.}}\]
{\bf Вязкое трение (сопротивление)} --- это трение, препятствующее движению тела
в сплошной среде. Сила вязкого трения пропорциональна скорости движения.
\[F_{\text{в.тр.}} = kv\]
\subsection{Электромагнитные силы}
{\bf Электрический заряд} --- это метра электрического взаимодействия тела. \\
{\bf Электрическое поле} --- это поле, созданное электрическими зарядами и
проявляющее себя действием на электрические заряды. \\
{\bf Напряжённость поля} --- это мера действия электрического поля на заряд. \\
\[\vv{E} = \frac{\vv{F}}{q}\]
{\bf Сила Кулона} --- это сила взаимодействия двух точечных зарядов в вакууме.
\\
\[\vv{F_q} = q\vv{E}\]
{\bf Электрический ток} --- это направленное движение заряженных частиц под
воздействием электрического поля. \\
{\bf Магнитное поле} --- это поле, созданное электрическим током и проявляющее
себя действием на движущиеся электрические заряды. \\
{\bf Магнитная индукция} --- это мера действия магнитного поля на заряд. \\
{\bf Электромагнитное поле} --- это поле, образованное электрическим и магнитным
полями, направленными перпендикулярно друг другу. \\
{\bf Сила Лоренца} --- это сила, с которое электромагнитное поле действует
движущийся точечный заряд.\\
\[F_L = q[\vv{v}\times \vv{B}]\]
\subsection{Релятивистское уравнение движения}
Обобщим второй закон Ньютона на случай движения тел с большими скоростями. Для
этого введём сопровождающую систему отсчёта, в которой выполняется второй закон
Ньютона, далее перейдём к неподвижной системе отсчёта с осями координат,
параллельными осям сопровождающей системы (используем преобразования Лоренца), а
затем поворачиваем неподвижную систему отсчёта.
\[\dot{\vv{p}} = \vv{F}\]
\[\vv{p} = \frac{m\vv{v}}{\sqrt{1-\frac{v^2}{c^2}}}\]
\section{Неинерциальные системы отсчёта. Сила инерции}
{\bf Инерциальная система отсчёта} --- это такая система, в которой любое тело,
бесконечно удалённое от других тел, не испытывает ускорения. Систему отсчёта,
связанную с Землёй, обычно можно считать инерциальной. Неинерциальными  являются
системы отсчёта, движущиеся с большим ускорением относительно Земли.\\
{\bf Сила инерции} --- добавочная сила, действующая на материальную точку в
неинерциальной системе отсчёта. Сила инерции отлична от нуля только для
наблюдателя, связанного с неинерциальной системой отсчёта, и не подчиняется
третьему закону Ньютона.\\
\[\vv{F} = m\vv{a}\]
\[\vv{F} = m\vv{a} + m\vv{a'} - m\vv{a'}\]
\[m\vv{a'} = \vv{F} - m(\vv{a} - \vv{a'})\]
\[\boxed{\vv{F_{\text{ин}}} = -m(\vv{a} - \vv{a'})}\]
\[\vv{a} = \vv{a'} + \vv{a_{\text{п}}} + \vv{a_{\text{к}}}\]
\[\vv{F_{\text{ин}}} = \vv{F_{\text{п}}} + \vv{F_{\text{к}}} =
-m\vv{a_0}+m\omega^2\vv{r} - 2m[\vv{\omega}\times \vv{v'}]\]
Второе слагаемое в этой сумме называется {\bf центробежной силой}, а третье ---
{\bf Кориолисовой силой}. \\
На тела, движущиеся в северном полушарии, действует сила Кориолиса, направленная
вправо относительно движения. Например, плоскость колебаний маятника Фуко
медленно поворачивается за счёт силы Кориолиса. Этот опыт доказывает вращение
Земли.\\
{\bf Невесомость} --- это исчезновение веса тела, вызванное ускорением системы
отсчёта. \\
{\bf Перегрузка} --- это возрастание веса тела, вызванное ускорением системы
отсчёта. \\
{\bf Центрифуга} --- это устройство, использующее центробежную силу инерции. \\

\section{Импульс системы частиц. Движение центра масс}
\subsection{Основные определения}
{\bf Импульс системы частиц} --- это сумма импульсов отдельных частиц системы.\\
\[\vv{p} = \sum_i\vv{p_i} = \sum_im_i\vv{v_i}\]
{\bf Центр масс системы частиц} --- это точка, радиус-вектор которой
определяется формулой:
\[\vv{r_c} = \frac{1}{m}\sum_im_i\vv{r_i}, \text{ где } m = \sum_im_i\]
Для однородных и симметричных тел центр масс совпадает с геометрическим центром.
В качестве примере рассмотрим систему из двух одинаковых точек.
\[\vv{r_c} = \frac{1}{2m}(\vv{r_1}m + \vv{r_2}m) = \frac{\vv{r_1}+\vv{r_2}}{2}\]
\subsection{Движение центра масс}
Импульс тела зависит от скорости центра масс
\[\begin{dcases}
	\vv{v_c} = \dot{r_c} = \frac{1}{m}\sum_im_i\vv{v_i} \\
	\vv{a_c} = \dot{v_c} = \frac{1}{m}\sum_im_i\vv{a_i} \\
\end{dcases}\]
\[\vv{p} = \sum_im_i\vv{v_i} = m\vv{v_c}\]

{\bf Внутренние силы} --- это силы взаимодействия между телами данной системы.
\\
\[\vv{f_{ij}} \text{ --- сила, действующая на i со стороны j }\]
{\bf Внешние силы} --- это силы, действующие на тела системы, со стороны тел, не
входящих в данную систему. \\
\[\vv{F_i} \text{ --- сила, действующая на i}\]
Просуммируем все силы
\[\sum_im_i\vv{a_i} = \sum_{i,j}\vv{f_{ij}} + \sum_i\vv{F_i}\]
По третьему закону Ньютона $\vv{f_{ij} = -\vv{f_{ji}}}$, то есть сумма
внутренних сил для любой пары частиц равна нулю. Следовательно, сумма всех
внутренних сил системы равна нулю
\[\sum_im_i\vv{a_i} = \sum_i\vv{F_i} = m\vv{a_c}\]
\[\boxed{m\vv{a_c} = \vv{F_{\text{внеш}}}}\]
Центр масс движется так, как если бы в нём находилась вся масса системы и к ней
были бы приложены все внешние силы.

\section{Закон сохранения импульса}
\subsection{Законы сохранения и изменения импульса}
{\bf Закон сохранения импульса}: если сумма внешних сил равна нулю, то импульс
системы сохраняется. \\
{\bf Закон изменения импульса}: изменение импульса равно сумме внешних сил,
действующих на систему. \\
\[\vv{p} = \sum_im_i\vv{v_i}\]
\[\dot{\vv{p}} = \sum_im_i\vv{a_i} = m\vv{a_c} =\vv{F_{\text{внеш}}} \]
{\bf Закон сохранения импульса}: если существует ось, проекция внешних сил на
которую равна нулю, то импульс системы относительно этой оси сохраняется. \\
\subsection{Реактивное движение}
Введём следующие обозначения: $\vv{v}$ --- скорость ракеты относительно Земли,
$\vv{u}$ --- скорость газов относительно Земли, $\vv{c}$ --- скорость газов
относительно ракеты, $\mu$ --- удельный расход топлива, $m$ --- масса ракеты с
топливом, $dm$ --- масса сгорающего топлива.
\[\begin{dcases}
	\vv{u} = \vv{v} + \vv{c} \\
	m\vv{v} = (m-dm)(\vv{v}+d\vv{v})+\vv{u}dm \\
\end{dcases}\]
\[md\vv{v} = -\vv{c}dm\]
\[m\vv{a} = -\vv{c}\frac{dm}{dt} = -\vv{c}\mu\]
\[\boxed{\vv{F} = -\mu\vv{c}}\]
\[m\frac{dv}{dt} = \mu c\]
\[\frac{dv}{c} = \frac{\mu dt}{m}\]
\[m(t) = m_0 - \mu t\]
\[\int_0^v\frac{dv}{c} = \int_0^t\frac{\mu dt}{m0-\mu t}\]
\[\boxed{\frac{v}{c} = \ln\frac{m_0}{m}}\]

\section{Работа и потенциальная энергия}
\subsection{Основные определения}
{\bf Элементарная работа} --- это скалярное произведение силы на бесконечно
малое перемещение точки приложения силы. \\
\[dA = (\vv{F}, d\vv{r})\]
{\bf 1 Джоуль} --- это работа, которую совершает сила в 1 Ньютон при
перемещении точки приложения на 1 метр в направлении силы. \\
{\bf Работа} --- это сумма элементарных работ. \\
\[A = \int dA\]
{\bf Потенциальная сила} --- это сила, работа которой равна нулю при перемещении
точки приложения по любому замкнутому контуру. Потенциальными являются,
например, сила тяжести, сила упругости, сила Кулона. Непотенциальной является
сила трения. \\
{\bf Элементарная потенциальная энергия} --- это элементарная работа
потенциальной силы, взятая со знаком минус.
\[d\Pi = -dA\]
{\bf Потенциальная энергия} --- это сумма элементарных потенциальных энергий. \\
\[\Pi = \int d\Pi\]
\subsection{Связь силы и потенциальной энергии}
\[d\Pi = -dA = -(\vv{F}, d\vv{r})\]
\[(\vv{F}, d\vv{r}) = F_xdx+F_dy+F_zdz = \frac{\partial \Pi}{\partial x}dx +
\frac{\partial \Pi}{\partial y}dy + \frac{\partial \Pi}{\partial z}dz\]
\[F = -\left( \vv{i}\frac{\partial \Pi}{\partial x} +
\vv{j}\frac{\partial \Pi}{\partial y} + \vv{z}\frac{\partial \Pi}{\partial z}
\right)\]
{\bf Потенциальная энергия системы частиц} --- это сумма потенциальных энергий
отдельных частиц системы.
\subsection{Потенциальная энергия материальной точки в поле центральной силы}
{\bf Центральная сила} --- это сила, направленная в сторону одной точки,
называемой силовым центром.
\[F = G\frac{Mm}{r^2}\]
\[d\Pi = -dA = -(\vv{F}, d\vv{r}) = \frac{F}{r}\vv{r}d\vv{r}\]
\[\vv{r}d\vv{r} = xdx + ydy + zdz = \frac{1}{2}d\left(x^2+y^2+z^2\right) =
\frac{1}{2}d\left(r^2\right) = rdr\]
\[d\Pi = Fdr\]
\[\Pi = GMm\int_{r_0}^r = GMm\left(\frac{1}{r} - \frac{1}{r_0}\right)\]

\section{Кинетическая энергия}
\subsection{Основные определения}
{\bf Кинетическая энергия материальной точки} --- это величина, определяемая
формулой:
\[K = \frac{mv^2}{2}\]
{\bf Закон изменения кинетической энергии}: приращение кинетической энергии
материальной точки равно работе действующей на неё силы.
\[m\frac{d\vv{v}}{dt} = \vv{F}\]
\[\vv{F}d\vv{r} = dA = m\frac{d\vv{r}}{dt}d\vv{v} =
d\left(\frac{mv^2}{2}\right) = dK\]
\[\boxed{dK = dA}\]
{\bf Кинетическая энергия системы частиц} --- это сумма кинетических энергий
отдельных частиц системы.
\[K = \sum_iK_i = \sum_i\frac{m_iv_i^2}{2}\]
\subsection{Кинетическая энергия твёрдого тела при поступательном движении}
\[K = \sum_iK_i = \sum_i\frac{m_iv_i^2}{2} = \frac{mv^2}{2}\]
\subsection{Кинетическая энергия твёрдого тела при вращении вокруг оси. Момент
инерции}
\[v_i = \omega r_i\]
\[K = \sum_iK_i = \sum_i\frac{m_iv_i^2}{2} = \frac{\omega^2}{2}\sum_im_ir_i^2\]
{\bf Момент инерции} --- это величина, определяемая формулой:
\[I = \sum_im_ir_i^2\]
\[K = \frac{I\omega^2}{2}\]
\subsection{Кинетическая энергия твёрдого тела при плоском движении}
{\bf Плоское движение} --- это движение, при котором все точки тела движутся
параллельно неподвижной плоскости.
\[\vv{r_i} = \vv{r_c} + \vv{r_{ic}}\]
\[\vv{v_i} = \vv{v_c} + \vv{v_{ic}}\]
\[K = \sum_i\frac{m_iv_i^2}{2} = \frac{1}{2}\sum_im_i\left(\vv{v_c} +
\vv{v_{ic}}\right)\left(\vv{v_c} + \vv{v_{ic}}\right) = \frac{1}{2}mv_c^2 +
\frac{1}{2}\sum_im_iv_{ic}^2 + \left(\vv{v_c}, \sum_im_i\vv{v_{ic}}\right)\]
\[\sum_im_i\vv{v_{ic}} = \sum_im_i\left(\vv{v_i} - \vv{v_c}\right) =
\sum_im_i\vv{v_{i}} - m\vv{v_c} = 0\]
\[\boxed{K = \frac{1}{2}mv_c^2 + \frac{1}{2}\sum_im_iv_{ic}^2}\]
Кинетическая энергия тела равна сумме кинетических энергий
движения центра и движения относительно центра.\\
{\bf Теорема Кёнига}: кинетическая энергия твёрдого тела при плоском движении
выражается формулой:
\[K = \frac{mv_c^2}{2} + \frac{I\omega^2}{2}\]
\section{Закон сохранения энергии в механике}
{\bf Полная механическая энергия системы} --- это сумма кинетической и
потенциальной энергий. \\
{\bf Закон сохранения механической энергии}: если работа непотенциальных сил
равна нулю, то полная механическая энергия системы сохраняется. \\
{\bf Закон изменения механической энергии}: приращение полной механической
энергии системы равно работе непотенциальных сил.
\[dK = dA = (\vv{F}, d\vv{r}) = (\vv{F_{\text{п}}}, d\vv{r}) +
(\vv{F_{\text{нп}}}, d\vv{r}) = dA_{\text{п}} + dA_{\text{нп}} = -d\Pi +
dA_{\text{нп}}\]
\[\boxed{d(K + \Pi) = dA_{\text{нп}}}\]
\section{Импульс и энергия в теории относительности}
\subsection{Законы сохранения импульса и энергии}
{\bf Релятивистские импульс и энергия} определяются формулами:
\[\vv{p} = \frac{m\vv{v}}{\sqrt{1-\frac{v^2}{c^2}}}\]
\[E = \frac{mc^2}{\sqrt{1-\frac{v^2}{c^2}}}\]
Импульс и энергия системы определяются как алгебраические суммы:
\[\vv{p} = \sum_i\vv{p_i}, E = \sum_iE_i\]
Если сумма внешних сил равна нулю, то релятивистский импульс и энергия
сохраняются. Эти два закона действуют в теории относительности действуют вместе.
Закон сохранения энергии является следствием закона сохранения импульса и
принципа относительности Эйнштейна. 
\subsection{Превращения массы и энергии}
Рассмотрим абсолютно неупругий удар двух одинаковых частиц.
\[\begin{dcases}
	\frac{m\vv{v}}{\sqrt{1-\frac{v^2}{c^2}}} -
	\frac{m\vv{v}}{\sqrt{1-\frac{v^2}{c^2}}} =
	\frac{M\vv{v'}}{\sqrt{1-\frac{v'^2}{c^2}}} \\
	\frac{mc^2}{\sqrt{1-\frac{v^2}{c^2}}} + \frac{mc^2}{\sqrt{1-\frac{v^2}{c^2}}}
	= \frac{Mc^2}{\sqrt{1-\frac{v'^2}{c^2}}}
\end{dcases}\]
\[\vv{v'} = 0\]
\[M = \frac{2m}{\sqrt{1-\frac{v^2}{c^2}}} > 2m\]
Это означает, что возможны взаимные превращения массы и энергии.
\subsection{Фотоны}
{\bf Фотон} --- это частица, энергия и импульс которой связаны соотношениями
$E=pc$ и $v=c$. Их можно получить из релятивистских уравнений энергии и
импульса, положив $m=0$. \\
Свет отказывает давление на поглощающие, отражающие и преломляющие его тела.
\[F = \dot{p} = \frac{\dot{E}}{c} = \frac{P}{c}\]

\section{Момент импульса частицы и системы частиц. Момент силы}
\subsection{Основные определения}
{\bf Момент импульса частицы} --- это векторное произведение радиус-вектора
частицы на её импульс.
\[\vv{N} = [\vv{r}\times\vv{p}]\]
{\bf Момент импульса системы частиц} --- это сумма моментов импульса отдельных
частиц.
\[\vv{N} = \sum_i\vv{N_i}\]
{\bf Момент силы} --- это векторное произведение радиус-вектора точки приложения
силы (плеча) на вектор силы.
\[\vv{M} = [\vv{r}\times\vv{F}]\]
{\bf Момент силы относительно оси} --- это проекция вектора момента силы на
данную ось.\\
Найдём момент силы относительно оси. Для этого разложим векторы силы и плеча на
составляющие, параллельные и перпендикулярные данной оси.
\[\vv{F} = \vv{F_{\parallel}} + \vv{F_{\perp}}\]
\[\vv{r} = \vv{r_{\parallel}} + \vv{r_{\perp}}\]
\[\vv{M} = \left[\vv{r_{\parallel}} + \vv{r_{\perp}}\times \vv{F_{\parallel}} +
\vv{F_{\perp}}\right] = \left[\vv{r_{\parallel}}\times\vv{F_{\parallel}}\right] +
\left[\vv{r_{\parallel}}\times\vv{F_{\perp}}\right] + \left[\vv{r_{\perp}}\times
\vv{F_{\parallel}}\right] + \left[\vv{r_{\perp}}\times\vv{F_{\perp}}\right]\]
\[\vv{M} = \vv{M_{\perp}} + \left[\vv{r_{\perp}}\times\vv{F_{\perp}}\right]\]
Чтобы найти составляющую момента силы, параллельную некоторой оси, нужно взять в
векторном произведении компоненты векторов, перпендикулярные этой оси.
Аналогичное утверждение верно и для момента импульса.\\
\subsection{Момент импульса материальной точки}
Момент импульса материальной точки равен произведению её импульса на расстояние
от линии движения до начала отсчёта.
\[M = Fr\sin\alpha = FR\]
\subsection{Момент импульса тела при вращении вокруг оси}
Найдём составляющую момента импульса, параллельную оси вращения.
\[\vv{N_{\parallel}} = \sum_i\left[\vv{r_{i\perp}}\times m\vv{v_i}\right]\]
\[\vv{v_i} = \left[\vv{\omega}\times\vv{r_{i\perp}}\right]\]
\[\vv{N_{\parallel}} = \sum_i\left[\vv{r_{i\perp}}\times
m\left[\vv{\omega}\times\vv{r_{i\perp}}\right]\right] = \sum_im_ir_{i\perp}^2\vv{\omega} =
\vv{\omega}I\]
\[\boxed{N_x = \omega_xI}\]
\subsection{Момент импульса при плоском движении тела}
\[\vv{r_i} = \vv{r_c} + \vv{r_{ic}}\]
\[\vv{v_i} = \vv{v_c} + \vv{v_{ic}}\]
\[\vv{N} = \sum_i\left[\vv{r_i}\times m_i\vv{v_i}\right] =
\sum_im_i\left[\vv{r_c} + \vv{r_ic} \times \vv{v_c}+\vv{v_{ic}}\right] =
\left[\vv{r_c}\times m\vv{v_c}\right] + \left[\vv{r_c} \times
\sum_im_i\vv{v_{ic}}\right] + \left[\sum_im_i\vv{r_{ic}} \times \vv{v_i}\right]
+ \sum_i\left[\vv{r_{ic}}\times m_i\vv{v_{ic}}\right]\]
\[\vv{r_c} = \frac{1}{m}\sum_im_i\vv{r_i}\]
\[\vv{v_c} = \frac{1}{m}\sum_im_i\vv{v_i}\]
\[\boxed{\vv{N} = \vv{N_c} + \vv{N_{oc}}}, \text{ где } \vv{N_c} =
\left[\vv{r_c}\times m\vv{v_c}\right], \vv{N_{oc}} =
\sum_i\left[\vv{r_{ic}}\times m_i\vv{v_{ic}}\right]\] 
\section{Теорема моментов. Закон сохранения момента импульса}
{\bf Теорема моментов}: скорость изменения момента импульса материальной точки
равна моменту действующей на неё силы.
\[\vv{N} = [\vv{r}\times m\vv{v}]\]
\[\vv{M} = [\vv{r}\times \vv{F}]\]
\[\dot{\vv{N}} = [\dot{\vv{r}}\times m\vv{v}] + [\vv{r}\times m\dot{\vv{v}}]\]
\[\boxed{\dot{\vv{N}} = \vv{M}}\]
{\bf Закон изменения момента импульса}: скорость изменения момента импульса
системы частиц равна сумме моментов внешних сил.\\
{\bf Закон сохранения момента импульса}: если сумма моментов внешних сил равна
нулю, то момент импульса системы сохраняется.
\[\vv{N} = \sum_i[\vv{r_i}\times m\vv{v_i}]\]
\[\dot{\vv{N}} = \sum_i\left[\dot{\vv{r_i}}\times m\vv{v_i}\right] +
\left[\vv{r_i}\times m\dot{\vv{v_i}}\right]\]
\[\vv{M_{ij}} = \left[\vv{r_i}\times \vv{f_{ij}}\right] + \left[\vv{r_j}\times
\vv{f_{ji}}\right] = \left[\vv{r_i} - \vv{r_j} \times \vv{f_{ij}}\right] = 0\]
\[\boxed{\dot{\vv{N}} = \vv{M_{\text{внеш}}}}\]

\section{Материальная точка в центральном поле}
\subsection{Движение в центральном поле}
{\bf Центральная сила} --- это сила, направленная в сторону одной точки,
называемой силовым центром. \\
{\bf Центральное поле} --- это поле центральной силы. \\
{\bf Секторная скорость} --- это скорость изменения площади, которую очерчивает
радиус-вектор точки при её движении по кривой.
\[\vv{M} = [\vv{r} \times \vv{F}] = 0\]
\[\vv{N} = [\vv{r} \times m\vv{v}] = const\]
При движении в центральном поле сохраняется момент импульса материальной точки.
Отсюда следует, что движение в центральном поле всегда является плоским, а
секторная скорость движения сохраняется.
\subsection{Законы Кеплера}
{\bf Первый закон Кеплера}: планеты Солнечной системы движутся по эллипсам, в
общем фокусе которых находится Солнце. \\
{\bf Второй закон Кеплера}: за равные промежутки времени радиус-вектор планеты
очерчивает равные площади. \\
{\bf Третий закон Кеплера}: квадраты периодов обращения планет относятся как
кубы больших полуосей их орбит. \\

\section{Плоское движение твёрдого тела}
Плоское движение твёрдого тела описывается двумя основными уравнениями ---
уравнением движения центра масс и уравнением вращения.
\[\begin{dcases}
	m\vv{a_c} = \vv{F_{\text{внеш}}} \\
	\dot{\vv{N}} = \vv{M}
\end{dcases} \]
{\bf Угловое ускорение} --- это производная угловой скорости по времени.
\[N_x = I\omega_x, \omega_x = \dot{\varphi}, \varepsilon_x = \dot{\omega_x} =
\ddot{\varphi}\]
\[\boxed{I\varepsilon_x = M_x}\]
Суммарный момент внешних сил относительно оси вращения равен произведению
момента инерции тела относительно этой оси на проекцию углового ускорения. Этим
уравнением можно пользоваться для любой неподвижной они и для подвижной оси,
проходящей через центр масс тела.\\
\subsection{Физический маятник}
{\bf Физический маятник} --- это тело произвольной формы, имеющее неподвижную
горизонтальную ось вращения. Найдём суммарный момент сил тяжести.
\[\vv{M} = \sum_i\left[\vv{r_i}\times m_i\vv{g}\right] =
\left[\sum_im_i\vv{r_i}\times \vv{g}\right] = \left[m\vv{r_c}\times
\vv{g}\right]\]
Суммарный момент сил тяжести, действующий на физический маятник, таков, как если
бы вся его масса находилась в центре масс.
\[\vv{M_{\parallel}} = \left[m\vv{r_{c\perp}}\times \vv{g}\right]\]
\[M_x = -M_{\parallel} = -mgl\sin\alpha\]
\[I\varepsilon_x = -mgl\sin\alpha\]
\[\boxed{\ddot{\varphi} + \frac{mgl}{I}\sin\alpha = 0}\]
\subsection{Цилиндр}
По наклонной плоскости без проскальзывания скатывается цилиндр.
\[m\ddot{x} = mg\sin\alpha-F_{\text{тр}}\]
\[I\varepsilon = F_{\text{тр}}R\]
\[\ddot{x} = \ddot{\varphi}R = \varepsilon R\]
\[F_{\text{тр}} = \frac{I\varepsilon}{R} = \frac{I\ddot{x}}{R^2}\]
\[m\ddot{x} = mg\sin\alpha - \frac{I\ddot{x}}{R^2}\]
\[\boxed{\ddot{x} = \frac{g\sin\alpha}{1+\frac{I}{mR^2}}}\]
\subsection{Сводка формул по плоскому движению твёрдого тела}
Движение центра масс
\[m\vv{a_c} = \vv{F_{\text{внеш}}}\]
Уравнение вращения
\[I\varepsilon_x = M_x\]
Импульс тела
\[\vv{p} = m\vv{v_c}\]
Кинетическая энергия и момент импульса при вращении вокруг неподвижной оси
\[K = \frac{I\omega^2}{2}\]
\[\vv{N_{\parallel} = I\vv{\omega}}\]
Кинетическая энергия и момент импульса при произвольном плоском движении
\[K = \frac{mv_c^2}{2} + \frac{I\omega^2}{2}\]
\[\vv{N_{\parallel}} = \left[\vv{r_{c\perp}}\times m\vv{v_c}\right] +
I\vv{\omega}\]

\section{Момент инерции тела}
\subsection{Теорема Гюйгенса-Штейнера}
{\bf Теорема Гюйгенса-Штейнера}: момент инерции тела относительно произвольной
оси равен сумме момента инерции относительно оси, проходящей через центр масс тела
параллельно данной, и полной массе тела, умноженной на квадрат расстояния между
осями.
\[I = I_c + ml^2\]
Для доказательства представим радиус-вектор точки как векторную сумму
радиус-векторов центра масс тела и точки относительно центра масс.
\[\vv{r_i} = \vv{r_c} + \vv{r_{ic}}\]
\[\vv{r_{i\perp}} = \vv{r_{c\perp}} + \vv{r_{ic\perp}}\]
\[I = \sum_im_ir_{i\perp}^2 = \sum_im_i\left(\vv{r_{i\perp}},
\vv{r_{i\perp}}\right) =
\sum_im_i\left(\vv{r_{c\perp}}+\vv{r_{ic\perp}},\vv{r_{c\perp}} +
\vv{r_{ic\perp}}\right) = \sum_i{m_i\left(r_{c\perp}^2 + 2\left(\vv{r_{c\perp}},
\vv{r_{ic\perp}}\right) + r_{ic\perp}^2\right)}\]
\[\sum_im_i\vv{r_{ic}} = \sum_im_i\left(\vv{r_i}-\vv{r_c}\right) =
\sum_im_i\vv{r_i} - m\vv{r_c} = 0\]
\[I = mr_{c\perp}^2 + \sum_im_ir_{ic\perp}^2\]
\[\boxed{I = ml^2 + I_c}\]
\subsection{Момент инерции стержня}
Представим стержень как совокупность материальных точек и найдём элементарные
моменты импульса.
\[I = I_c + ml^2\]
\[dI = l^2dm\]
\[dm = \frac{m}{L}dl\]
\[dI = \frac{m}{L}l^2dl\]
Если ось проходит через центр:
\[I = \frac{m}{L}\int_{-\frac{L}{2}}^{\frac{L}{2}}l^2dl\]
\[\boxed{I = \frac{mL^2}{12}}\]
Если ось проходит через край:
\[I = \frac{mL^2}{12} + \frac{mL^2}{4} = \frac{mL^2}{3}\]
\subsection{Момент инерции диска}
\[dI = r^2dm\]
\[dm = m\frac{dS}{S}\]
\[dS = rdrd\varphi\]
\[dI = r^2m\frac{rdrd\varphi}{S}\]
\[I = \frac{m}{S}\int_0^{2\pi}d\varphi\int_0^Rr^3dr = \frac{\pi mR^4}{2S}\]
\[S = \pi R^2\]
\[\boxed{I = \frac{mR^2}{2}}\]
\section{Системы со связями. Степени свободы. Обобщённые координаты}
\subsection{Связи в механике}
{\bf Связи в механике} --- это не вытекающие из уравнений движения ограничения
на координаты, скорости и ускорения точек системы. В системах со связями
действуют заданные силы и силы реакции. \\
{\bf Заданные силы} --- это известные постоянные силы или известные функции
координат и скоростей частиц. \\
{\bf Силы реакции} --- это силы, которые действуют на тела системы со стороны
тел, осуществляющих связи. \\
{\bf Голономные связи} --- это связи, которые сводятся к ограничениям только на
координаты тел. Остальные связи называются неголономными. \\
{\bf Стационарные связи} --- это связи, уравнения которых не содержат времени в
явном виде. Остальные связи называются нестационарными. Далее мы будем
рассматривать только голономные стационарные связи. \\
\subsection{Степени свободны и обобщённые координаты}
{\bf Число степеней свободы} --- это число независимых координат, полностью
определяющих положение системы в пространстве. Обозначим число степеней свободы
через $s$.\\
{\bf Обобщённые координаты} --- это любые $s$ координат, полностью определяющие
положение системы в пространстве. Обозначим обобщённые координаты через ${q}$.
Обобщённые координаты обращают в тожества уравнения связей. Радиус-векторы точек
системы являются функциями обобщённых координат. \\
{\bf Обобщённые скорости} --- это производные по времени обобщённых координат.\\

\section{Виртуальные перемещения. Виртуальная работа. Идеальные связи}
{\bf Виртуальное перемещение} --- это воображаемое бесконечно малое перемещение
точки, допускаемое связями в данный момент времени.
\[\vv{r_l} = \vv{r_l}(q_1,\dots,q_s, t)\]
\[\delta\vv{r_l} = \sum_{i=1}^s\frac{\partial\vv{r_l}}{\partial q_i}\delta q_i\]
{\bf Виртуальная работа} --- работа силы на виртуальном перемещении.
\[\delta A = \left(\vv{F}, \delta\vv{r}\right)\]
{\bf Идеальные связи} --- это связи, для которых виртуальная работа сил реакции
равна нулю.
\[\delta A_R = \sum_l\left(\vv{R_l}, \delta\vv{r_l}\right) = 0\]

\section{Уравнения Лагранжа. Обобщённые силы}
Будем рассматривать систему с идеальными голономными связями. Выведем
дифференциальное уравнение второго порядка, описывающее её движение через
обобщённые координаты.
\[\vv{r_l} = \vv{r_l}(q_1,\dots, q_s, t)\]
\[\delta\vv{r_l} = \sum_{i=1}^s\frac{\partial\vv{r_l}}{\partial q_i}\delta q_i\]
\[\delta A_R = \sum_l\left(\vv{R_l}, \delta\vv{r_l}\right) = 0 \text{ (поскольку
связи идеальные)}\]
\[m_l\ddot{\vv{r_l}} = \vv{F_l} + \vv{R_l}\]
\[\boxed{\sum_l\left(m_l\ddot{\vv{r_l}}-\vv{F_l}, \delta\vv{r_l}\right) = 0}\]
\[\sum_l\left(m_l\ddot{\vv{r_l}}-\vv{F_l}, \sum_{i=1}^s\frac{\partial\vv{r_l}}
{\partial q_i}\delta q_i \right) = 0 \text{ (подставили выражение для }
\delta\vv{r_l})\]
\[\sum_{i=1}^s\delta q_i\sum_l\left(m_l\ddot{\vv{r_l}}-\vv{F_l},
\frac{\partial\vv{r_l}}{\partial q_i}\right) = 0 \text{ (вынесли скаляр,
изменили порядок суммирования)}\]
Обозначим $X_i = \sum_l\left(m_l\ddot{\vv{r_l}}, \frac{\partial\vv{r_l}}
{\partial q_i}\right)$, $Q_i = \sum_l\left(\vv{F_l}, \frac{\partial\vv{r_l}}
{\partial q_i}\right)$.
\[\sum_{i=1}^s\left(X_i-Q_i\right)\delta q_i = 0\]
\[X_i = Q_i\]
Покажем, что $X_i$ можно выразить через кинетическую энергию системы и её
производные по обобщённым координатам и скоростям.
\[K = \frac{1}{2}\sum_lm_l\left(\vv{v_l},\vv{v_l}\right)\]
\[\vv{r_l} = \vv{r_l}(q_1,\dots, q_s, t)\]
\[\vv{v_l} = \dot{\vv{r_l}} = \sum_{i=1}^s\frac{\partial\vv{r_l}}{\partial
q_i}\dot{q_i} + \frac{\partial\vv{r_l}}{\partial t} = \vv{v_l}(q, \dot{q}, t)\]
\[\frac{\partial\vv{v_l}}{\partial \dot{q_i}} = \frac{\partial\vv{r_l}}{\partial
q_i}\]
\[\frac{\partial K}{\partial \dot{q_i}} = \frac{\partial}{\partial \dot{q_i}}
\left(\frac{1}{2}\sum_lm_l\left(\vv{v_l},\vv{v_l}\right)\right) =
\sum_lm_l\left(\vv{v_l},\frac{\partial\vv{v_l}}{\partial \dot{q_i}}\right) =
\sum_lm_l\left(\vv{v_l},\frac{\partial\vv{r_l}}{\partial q_i}\right)\]
\[\frac{d}{dt}\left(\frac{\partial K}{\partial \dot{q_i}}\right) = 
\sum_lm_l\left(\dot{\vv{v_l}},\frac{\partial\vv{r_l}}{\partial q_i}\right) + 
\sum_lm_l\left(\vv{v_l},\frac{\partial\vv{v_l}}{\partial q_i}\right) = X_i +
\frac{\partial K}{\partial q_i}\]
\[\boxed{X_i = Q_i = \frac{d}{dt}\left(\frac{\partial K}{\partial
\dot{q_i}}\right) - \frac{\partial K}{\partial q_i}}\]
Получили уравнение Лагранжа относительно кинетической энергии.\\
Величина $Q_i = \sum_l\vv{F_l}\frac{\partial\vv{r_l}}{\partial q_i}$ называется
{\bf обобщённой силой}. По своему физическому смыслу она представляет собой
проекцию силы на ось либо момент силы относительно оси.

\section{Функция Лагранжа. Обобщённые импульсы}
Рассмотрим систему с идеальными голономными связями и потенциальными заданными
силами. Получим функцию, выражающую разность кинетической и потенциальной
энергий системы через обобщённые координаты и скорости.
\[\vv{F_l} = -\left(\vv{i}\frac{\partial\Pi_l}{\partial x_l} +
\vv{j}\frac{\partial\Pi_l}{\partial y_l} +
\vv{k}\frac{\partial\Pi_l}{\partial z_l}\right)\]
\[\vv{F_l}\frac{\partial\vv{r_l}}{\partial q_i} = -\left(
\frac{\partial\Pi_l}{\partial x_l} \frac{\partial x_l}{\partial q_i} +
\frac{\partial\Pi_l}{\partial y_l} \frac{\partial y_l}{\partial q_i} +
\frac{\partial\Pi_l}{\partial z_l} \frac{\partial z_l}{\partial q_i}\right) =
-\frac{\partial\Pi_l}{\partial q_i}\]
Суммируя по всем $l$, получим:
\[Q_i = -\frac{\partial \Pi}{\partial q_i}\]
Подставим в уравнение Лагранжа относительно кинетической энергии:
\[-\frac{\partial \Pi}{\partial q_i} = \frac{d}{dt}\left(\frac{\partial
K}{\partial \dot{q_i}}\right) - \frac{\partial K}{\partial q_i}\]
{\bf Лагранжиан системы} --- это функция $L=K-\Pi = L(q, \dot{q}, t)$
--- разность кинетической и потенциальной энергий системы, выраженная через
обобщённые координаты, обобщённые скорости и время.
\[\boxed{\frac{d}{dt}\left(\frac{\partial L}{\partial\dot{q_i}}\right) = 0}\]
Получили уравнения Лагранжа относительно функции Лагранжа. \\
Величина $p_i = \frac{\partial L}{\partial\dot{q_i}}$ называется {\bf обобщённым
импульсом}. По своему физическому смыслу она представляет собой проекцию
импульса на ось либо момент импульса относительно оси.
{\bf Закон изменения обобщённого импульса}: скорость изменения обобщённого
импульса равна частной производной функции Лагранжа по соответствующей
обобщённой координате.
\[\dot{p_i} = \frac{\partial L}{\partial q_i}\]
{\bf Циклическая обобщённая координата} --- это такая обобщённая координата,
которая не входит явно в функцию Лагранжа. \\
{\bf Закон сохранения обобщённого импульса}: обобщённый импульс, соответствующий
циклической обобщённой координате, сохраняется.

\section{Уравнения Гамильтона. Канонические переменные}
{\bf Канонические переменные} --- это обобщённые координаты и импульсы
системы.\\
{\bf Уравнения Гамильтона} --- это уравнения движения, записанные в канонических
переменных. \\
Рассмотрим систему с идеальными голономными связями и потенциальными заданными
силами. Получим для неё уравнения Гамильтона.
\[\frac{d}{dt}\left(\frac{\partial L}{\partial\dot{q_i}}\right) = 0\]
\[dL = \sum_{i=1}^s\left(\frac{\partial L}{\partial q_i}dq_i + \frac{\partial L}
{\partial \dot{q_i}}d\dot{q_i}\right) + \frac{\partial L}{\partial t}dt =
\sum_{i=1}^s\left(\dot{p_i}dq_i + p_id\dot{q_i}\right) + \frac{\partial
L}{\partial t}dt \]
Обозначим $H = \sum_{i=1}^s\dot{q_i}p_i-L$
\[dH = \sum_{i=1}^s\left(\dot{q_i}dp_i + p_i\dot{q_i}\right) - dL =
\sum_{i=1}^s\left(\dot{q_i}dp_i + p_i\dot{q_i} - \dot{p_i}dq_i -
p_id\dot{q_i}\right) - \frac{\partial L}{\partial t}dt =
\sum_{i=1}^s\left(\dot{q_i}dp_i - \dot{p_i}dq_i \right) - \frac{\partial
L}{\partial t}dt\]
Продифференцируем $H = H(q, p, t)$ как функцию канонических переменных
\[dH = \sum_{i=1}^s\left(\frac{\partial H}{\partial q_i}dq_i + \frac{\partial
H}{\partial p_i}dp_i\right) + \frac{\partial H}{\partial t}dt\]
\[\boxed{\dot{q_i} = \frac{\partial H}{\partial p_i}, \dot{p_i} =
-\frac{\partial H}{\partial p_i}}\]
Получили уравнения Гамильтона или {\bf Гамильтониан} системы.

\section{Гамильтониан консервативной системы}
Дифференцируя функцию Гамильтона по времени, получим {\bf закон изменения
Гамильтониана}
\[\frac{dH}{dt} = \sum_{i=1}^s\left(\frac{\partial H}{\partial q_i}\dot{q_i} +
\frac{\partial H}{\partial p_i}\dot{p_i}\right) + \frac{\partial H}{\partial
t} = \sum_{i=1}^s\left(-\dot{p_i}\dot{q_i} + \dot{q_i}\dot{p_i}\right) +
\frac{\partial H}{\partial t}\]
\[\boxed{\frac{dH}{dt} = \frac{\partial H}{\partial t}}\]
{\bf Консервативная система} --- это система, Гамильтониан которой не зависит
явно от времени.\\
Покажем, что Гамильтониан консервативной системы имеет смысл её полной
механической энергии.
\[\frac{dH}{dt} = 0\]
\[H = \sum_{i=1}^s\dot{q_i}p_i - L\]
\[p_i = \frac{\partial L}{\partial \dot{q_i}} = \frac{\partial (K-\Pi)}{\partial
\dot{q_i}} = \frac{\partial K}{\partial \dot{q_i}}\]
\[K = \frac{1}{2}\sum_lm_l\left(\vv{v_l},\vv{v_l}\right)\]
\[\vv{v_l} = \dot{\vv{r_l}}\]
\[\vv{r_l} = \vv{r_l}(q_1,\dots,q_s,t)\]
\[\vv{v_l} = \sum_{i=1}^s\frac{\partial \vv{r_l}}{\partial q_i}\dot{q_i}\]
\[K = \frac{1}{2}\sum_lm_l\left(\sum_{a=1}^s\frac{\partial \vv{r_l}}{\partial
q_a}\dot{q_a}, \sum_{b=1}^s\frac{\partial \vv{r_l}}{\partial
q_b}\dot{q_b}\vv{v_l}\right) = \frac{1}{2}\sum_{a=1}^s\sum_{b=1}^s\dot{q_a}
\dot{q_b}\sum_lm_l \left(\frac{\partial \vv{r_l}}{\partial q_a}, \frac{\partial
\vv{r_l}}{\partial q_b}\right)\]
Обозначим $K_{ab}(q) = \sum_lm_l \left(\frac{\partial \vv{r_l}}{\partial q_a},
\frac{\partial \vv{r_l}}{\partial q_b}\right)$
\[K = \frac{1}{2} \sum_{a=1}^s\sum_{b=1}^s\dot{q_a}\dot{q_b}K_{ab}(q)\]
\[p_i = \frac{\partial K}{\partial \dot{q_i}} = \frac{1}{2}
\sum_{a=1}^s\sum_{b=1}^s\left(\dot{q_a}\frac{\partial
\dot{q_b}}{\partial \dot{q_i}} + \dot{q_b}\frac{\partial \dot{q_a}}{\partial
\dot{q_i}}\right) K_{ab}(q)\]
\[p_j = \frac{1}{2} \sum_{a=1}^s\dot{q_a}\sum_{b=1}^sK_{ab}\delta_{bi} +
\frac{1}{2} \sum_{b=1}^s\dot{q_b}\sum_{a=1}^sK_{ab}\delta_{ai}, \text{ где }
\delta_{jk} \text{ --- символ Кронекера}\]
\[\boxed{p_i = \sum_{j=1}^sK_{ij}\dot{q_j}}\]
Подставим полученное выражение для импульса в функцию Гамильтона
\[H = \sum_{i=1}^s\dot{q_i}\sum_{j=1}^sK_{ij}\dot{q_j} - L = \sum_{i=1}^s
\sum_{j=1}^sK_{ij}\dot{q_i}\dot{q_j}-L = 2K - L = K + \Pi\]
\[\boxed{H = K + \Pi}\]

\section{Равновесие системы и его устойчивость}
{\bf Равновесие} --- состояние, в котором система, предоставленная самой себе,
может находиться сколь угодно долго.\\
{\bf Условие равновесия}: сумма заданных сил и сил реакции, действующих на
каждую материальную точку тела, равна нулю.
\[F_l + R_l = 0\]
\[\delta A_R = \sum_l\left(\vv{R_l}, \delta\vv{r_l}\right) = 0\]
{\bf Принцип виртуального перемещения}: в состоянии равновесия виртуальная
работа заданных сил равна нулю.
\[\delta A_F = \sum_l\left(\vv{F_l}, \delta\vv{r_l}\right) = 0\]
\[\delta \vv{r_l} = \sum_{i=1}^s \frac{\partial \vv{r_l}}{\partial q_i}\delta
q_i\]
\[\sum_l\left(\vv{F_l}, \sum_{i=1}^s \frac{\partial \vv{r_l}}{\partial q_i}\delta
q_i\ \right) = \sum_{i=1}^s\delta q_i \left(\vv{F_l}, \sum_l \frac{\partial \vv{r_l}}{\partial q_i}\right) = 0\]
\[\sum_{i=1}^sQ_i\delta q_i = 0\]
Из независимости обобщённых координат имеем $Q_i = -\frac{\partial \Pi}{\partial
q_i} = 0$, то есть в состоянии равновесия потенциальная энергия испытывает
экстремум. \\
Равновесие называется {\bf устойчивым}, если система, выведенная из положения
равновесия и предоставленная самой себе, начинает двигаться в сторону положения
равновесия. Устойчивость определяется типом экстремума потенциальной энергии:
положение максимума устойчиво, а минимума --- нет.

\section{Колебания в системах с одной степенью свободы}
\subsection{Основные определения}
{\bf Колебания} --- это повторяющиеся движения в окрестности положения устойчивого
равновесия системы. \\
{\bf Гармоническая функция} --- это функция, меняющаяся по закону синуса или
косинуса.\\
{\bf Амплитуда колебаний} --- это максимальное отклонение колеблющегося тела от
положения равновесия. \\
{\bf Период колебаний} --- это время одного полного колебания. \\
{\bf Фаза колебаний} --- это аргумент тригонометрической функции, описывающей
колебания. \\
Уравнение гармонических колебаний:
\[\ddot{x} +\omega^2 x = 0\]
Общее решение:
\[x(t) = A\cos{(\omega t + \varphi)}\]
Найдём амплитуду и фазу колебаний
\[\begin{dcases}
	x(t = 0) = x_0 = A\cos{\varphi} \\
	\dot{x}(t = 0) = \dot{x_0} = -\omega A\sin{\varphi}
	\end{dcases}\]
Разделив второе уравнение на $\omega$, возведя в квадрат и сложив, получим:
\[A = \sqrt{x_0^2 + \frac{\dot{x_0}^2}{\omega^2}}\]
\[\sin\varphi = \frac{x_0}{A}\]
\[\cos\varphi = -\frac{\dot{x_0}}{\omega A}\]
Чтобы найти $\omega$, нужно записать уравнение движения и привести его к
стандартному виду $\ddot{x} +\omega^2 x = 0$.
\subsection{Физический маятник}
Найдём частоту колебаний физического маятника
\[s = 1, q = \varphi, \dot{q} = \omega\]
\[K = \frac{I\omega^2}{2} = \frac{I\dot{q}^2}{2}\]
\[\Pi = -mgl\cos\varphi\]
\[L = K-\Pi = \frac{I\dot{q}^2}{2} + mgl\cos{q}\]
\[\frac{d}{dt}\left(\frac{\partial L}{\partial \dot{q}}\right) - \frac{\partial
L}{\partial q} = 0\]
\[I\ddot{\varphi} + mgl\sin\varphi\]
\[\sin\varphi \approx \varphi\]
\[\ddot{\varphi} + \frac{mgl}{I}\varphi = 0\]
\[\boxed{\omega = \sqrt{\frac{mgl}{I}}}\]
Частным случаем физического маятника является математический маятник
\[\omega = \sqrt{\frac{mgl}{ml^2}} = \sqrt{\frac{g}{l}}\]
Другой частный случай --- маятник в виде кольца
\[\omega = \sqrt{\frac{mgl}{2ml^2}} = \sqrt{\frac{g}{2l}}\]
\subsection{Пружинный осциллятор}
Найдём частоту колебаний пружинного осциллятора
\[m\ddot{x} = mg - k(x-l_0)\]
\[m\ddot{x} + k\left(x-l_0-\frac{m}{k}\right) = 0\]
Обозначим $y = \left(x-l_0-\frac{m}{k}\right)$, тогда $\ddot{y} = \ddot{x}$
\[\ddot{y} + \frac{k}{m}y = 0\]
\[\boxed{\omega = \sqrt{\frac{k}{m}}}\]

\section{Физические эффекты в колебательных системах}
\subsection{Затухающие колебания}
Уравнение затухающих колебаний:
\[\ddot{x} + \alpha\dot{x} + \omega^2x = 0\]
{\bf Добротность колебательной системы} --- количество колебаний, которое будет
совершено до остановки маятника. Чем выше добротность, тем ближе затухающие
колебания к свободным гармоническим.
\[Q=\frac{T_{\text{зат}}}{T}\]
\subsection{Нелинейные и параметрические колебания}
Уравнение нелинейных колебаний:
\[\ddot{x} + \omega^2f(x) = 0\]
Основная особенность таких колебаний --- зависимость периода от амплитуды. \\
{\bf Параметрические колебания} --- это колебания, возникающие при изменении
какого-либо параметра колебательной системы в результате внешнего
взаимодействия. \\
{\bf Параметрический резонанс} --- это явление возрастания амплитуды колебаний в
результате параметрического возбуждения, создаваемого временным изменением
параметров системы. \\
Характерным примером параметрических колебаний являются качели.
\subsection{Вынужденные колебания}
{\bf Вынужденные колебания} --- это колебания, происходящие под действием
переменной внешней силы. \\
{\bf Резонанс} --- это явление возрастания амплитуды колебаний при совпадении
частоты изменения внешней силы с собственной частотой колебаний системы. \\

\section{Нормальные колебания и нормальные координаты}
{\bf Нормальные координаты} --- это обобщённые координаты, которые при любых
движениях системы меняются независимо друг от друга. Из независимости нормальных
координат следует, что возможны колебания, при которых только одна нормальная
координата отлична от нуля. Такие колебания называются {\bf нормальными}.
Нормальные колебания характеризуются частотой. \\
{\bf Нормальная (собственная) частота} --- это частота нормальных колебаний.\\
В качестве примера рассмотрим систему из двух пружинных осцилляторов.
Понятия нормальных координат и нормальных колебаний имеют смысл для систем с
любым числом степеней свободы.
\[\begin{dcases}
	m\ddot{x_1} = -kx_1 + k(x_2-x_1) \\
	m\ddot{x_2} = -kx_2 - k(x_2-x_1) 
\end{dcases}\]
\[\begin{dcases}
	m(\ddot{x_1} + \ddot{x_2}) = -k(x_1+x_2) \\
	m(\ddot{x_2} - \ddot{x_1}) = -3k(x_2-x_1) 
\end{dcases}\]
Введём новые координаты
\[\begin{dcases}
	y_1 = x_1 + x_2 \\
	y_2 = x_2 - x_1
\end{dcases}\]
\[\begin{dcases}
	\ddot{y_1} + \omega_1^2y_1 = 0, \omega_1 = \sqrt{\frac{k}{m}} \\
	\ddot{y_2} + \omega_2^2y_2 = 0, \omega_2 = \sqrt{\frac{3k}{m}}
\end{dcases}\]

\section{Колебания струны}
\subsection{Волновое уравнение}
{\bf Волна} --- это колебание, распространяющееся в сплошной среде. \\
Выведем волновое уравнение.
\[dm \frac{\partial^2 y}{\partial t^2} = T\sin\beta - T\sin\alpha\]
\[\sin\alpha \approx \tg\alpha = y'(x), \sin\beta \approx \tg\beta = y'(x+dx)\]
\[dm \frac{\partial^2 y}{\partial t^2} = T(y'(x+dx) - y'(x)) = Ty''(x)dx\]
Введём линейную плотность $\rho = \frac{m}{l}$
\[dm \approx \rho dx\]
\[\rho \frac{\partial^2 y}{\partial t^2} = T\frac{\partial^2 y}{\partial x^2}\]
\[\frac{T}{\rho} = v^2\]
\[\boxed{\frac{\partial^2 y}{\partial t^2} = v^2\frac{\partial^2 y}{\partial
x^2}}\]
Общее решение волнового уравнения:
\[y(x, t) = f_1(x-vt) + f_2(x+vt)\]
\subsection{Стоячие волны}
{\bf Стоячая волна} --- это волна, которая описывается произведением функции
времени на функцию координаты.
\[y(x, t) = P(x)Q(t)\]
\[P(x)\ddot{Q}(t) = v^2Q(t)P''(t)\]
\[\frac{\ddot{Q}(t)}{Q(t)} = v^2\frac{P''(t)}{P(t)} = -\omega^2\]
\[\begin{dcases}
	\ddot{Q}(t) + \omega^2Q(t) = 0 \\
	P''(t) + \frac{\omega^2}{v^2}P(t) = 0
\end{dcases}\]
Частное решение:
\[\begin{dcases}
	Q(t) = A\sin\omega t \\
	P(t) = B\sin\frac{\omega}{v}t
\end{dcases}\]

\section{Случайные величины и вероятности}
{\bf Случайное событие} --- это событие, исход которого нельзя предсказать, но
наблюдение которого можно многократно повторить. \\
{\bf Вероятность случайного события} --- отношение числа появления события в
серии испытаний в пределе при числе испытаний, стремящемся к бесконечности. \\
\[\mathbb{P}(A) = \lim_{N \to \infty} \frac{N_A}{N}\]
{\bf Аксиома сложения вероятностей}: вероятность наступления одного из
взаимоисключающих событий равна сумме их вероятностей. \\
\[\mathbb{P}(A\cup B) = \mathbb{P}(A) + \mathbb{P}(B)\]
{\bf Аксиома умножения вероятностей}: вероятность наступления независимых
случайных событий равна произведению их вероятностей.
\[\mathbb{P}(AB) = \mathbb{P}(A) \mathbb{P}(B)\]
{\bf Случайная величина} --- это величина, значение которой нельзя предсказать,
но измерение которой можно многократно повторить. \\
{\bf Распределение плотности вероятности} --- отношение вероятности попадания
непрерывной случайной величины в малый интервал вблизи заданного значения к
величине этого интервала в пределе при длине интервала, стремящейся к нулю.
\[\omega(x_0) = \lim_{\Delta x \to 0} \frac{\mathbb{P}(x_0 \leq x \leq
x_0+\Delta x)}{\Delta x}\]
{\bf Правило вычисления средних (математического ожидания)}:
\[\mathbb{E}f(x) = \overline{f(x)} = \int_{-\infty}^{\infty}f(x)\omega(x)dx\]
{\bf Дисперсия случайной величины} --- это средний квадрат её отклонения от
среднего значения.
\[\mathbf{D}x = \overline{(x-\overline{x})^2} = \overline{x^2} -
\overline{x}^2\]
{\bf Многомерная плотность вероятности} --- это отношение вероятности попадания
нескольких случайных величин в малые интервалы вблизи заданных значений к
произведению величин интервалов в пределе при длинах интервалов, стремящихся к
нулю.
\[\omega(x_0, y_0) = \lim_{\substack{\Delta x \to 0\\ \Delta y \to 0}}
\frac{\mathbb{P}(x_0 \leq x \leq x_0+\Delta x, y_0 \leq y \leq y_0+\Delta
y)}{\Delta x\Delta y}\]
Для независимых случайных величин многомерная плотность равна произведению
одномерных.

\section{Распределение Гиббса}
{\bf Основной закон статистической механики равновесных систем}: в состоянии
термодинамического равновесия распределение плотности вероятности для различных
состояний системы определяется {\bf формулой Гиббса}.
\[w(z) = Ce^{-\frac{H(z)}{kT}}\]
Здесь $z$ --- набор обобщённых координат и импульсов, $H(z)$ --- Гамильтониан
системы, $T$ --- абсолютная температура системы, $k \approx 1.38\cdot10^{-23}$
--- постоянная Больцмана. \\
{\bf Термодинамическое равновесие} --- состояние, в котором система,
предоставленная самой себе, может находиться сколь угодно долго.\\
{\bf Градус Цельсия} --- одна сотая часть интервала между температурой плавления
льда и температурой кипения воды при нормальных условиях. \\
Распределение молекул по скоростям (распределение Максвела):
\[\omega(v_x, v_y, v_z) = Ce^{-\frac{m(v_x^2+v_y^2+v_z^2)}{2kT}}\]
Распределение частиц в силовом поле (распределение Больцмана):
\[\omega(x, y, z) = Ce^{-\frac{\Pi(x, y, z)}{kT}}\]

\section{Размер и масса молекул}
{\bf Идеальный газ} --- это газ, силы взаимодействия между молекулами которого
пренебрежимо малы. \\
{\bf Концентрация молекул} --- это отношение числа молекул к объёму сосуда, в
котором они находятся. \\
\[n = \frac{N}{V}\]
\[dp = (2mv_x)(Sv_xdt)(n\omega(v_x)dv_x)\frac{1}{Sdt} =
2mnv_x^2\omega(v_x)dv_x\]
\[p = 2mn\int_0^{+infty}v_x^2\omega(v_x)dv_x = mn\int_{-\infty}^{+infty}
v_x^2\omega(v_x)dv_x = mn\mathbb{D}v_x\]
\[\boxed{p=nkt}\]

\section{Распределение энергии по степеням свободы}
{\bf Квадратичная степень свободы} --- переменная, вклад которой в Гамильтониан
системы пропорционален квадрату этой переменной.\\
{\bf Закон равнораспределения энергии по степеням свободы}: в состоянии
термодинамического равновесия на каждую квадратичную степень свободы приходится
одинаковое количество энергии: $E = \frac{kT}{2}$.
{\bf Температура} --- это мера средней кинетической энергии частиц, составляющих
систему. \\
{\bf Внутренняя энергия} --- это энергия движения и взаимодействия частиц,
составляющих систему. \\

\section{Диффузия и теплопроводность}
{\bf Диффузия} --- это проникновение одного вещества в другое. \\
{\bf Закон диффузии}: плотность потока частиц пропорциональна градиенту их
концентрации. \\
\[\j_x = -D\frac{\partial n}{\partial x}\]
{\bf Коэффициент диффузии} --- это коэффициент пропорциональности между
плотностью потока частиц и градиентом их концентрации. \\
Уравнение диффузии:
\[\frac{\partial n}{\partial t} = D\frac{\partial^2 n}{\partial x^2}\]
{\bf Теплопроводность} --- это процесс переноса тепла в однородном нагретом
теле. \\
{\bf Закон теплопроводности}: плотность потока тепла пропорциональна градиенту
температуры.
\[j_x = -\varkappa\frac{\partial T}{\partial x}\]
{\bf Коэффициент теплопроводности} --- это коэффициент пропорциональности между
плотностью потока тепла и градиентом температуры. \\
Уравнение теплопроводности:
\[\frac{\partial T}{\partial t} -
\frac{\varkappa}{c\rho} \frac{\partial^2 T}{\partial t^2} = 0\]

\section{Вязкость жидкости}
{\bf Вязкость} --- свойство жидкости препятствовать движению соприкасающихся с
ней тел. \\
{\bf Закон вязкости}: плотность потока импульса пропорциональна градиенту
скорости.
\[\frac{dp_x}{Sdt} = -\eta\frac{\partial v_x}{\partial y}\]
{\bf Коэффициент вязкости} --- это коэффициент пропорциональности между
плотностью потока импульса и градиентом скорости. \\
Сила вязкого трения:
\[F = \eta S \frac{v}{h}\]

\section{Движение вязкой жидкости}
Уравнение движения вязкой жидкости:
\[\rho\frac{dv_x}{dt} = \eta\frac{\partial^2 v_x}{\partial y^2}\]
{\bf Закон Пуазейля}: расход жидкости прямо пропорционален плотности жидкости,
разности давлений на входе и на выходе трубы и четвёртой степени её радиуса;
обратно пропорционален коэффициенту вязкости и длине трубы.
\[Q=t\frac{\pi \Delta p}{8\eta l}R^4\]

\section{Уравнения динамики сплошной среды}
Уравнение движения элемента сплошной среды:
\[\rho\frac{\partial v_x}{\partial t} + \frac{\partial p}{\partial x} = 0\]
\section{Звуковая волна}
{\bf Звуковые волны} --- это колебания вещества, передающиеся в сплошной среде.
Звуковые волны --- продольные. \\
{\bf уравнение состояния среды} --- это зависимость давления среды от плотности.
\[\frac{\partial^2v_x}{\partial t^2} = -c_0^2\frac{\partial^2v_x}{\partial
x^2}\]

\end{document}
